\documentclass{beamer}
\usepackage[spanish]{babel}
\usepackage{array}
\usepackage{comment}
\usepackage{color}
\usepackage{listings}
\usepackage{hyperref}
\usepackage{pgfpages}
\usepackage{pifont}
\usepackage{ulem}\normalem

\usepackage{../../../common/beamer-style/explicacion-practica}
\setbeamerfont{subsection in toc}{size=\small}

%Kernel values
\newcommand{\KERNELBASEVERSION}{3.13}
\newcommand{\KERNELVERSION}{\KERNELBASEVERSION.1}
\newcommand{\PATCHEDKERNELVERSION}{\KERNELBASEVERSION.3}
\newcommand{\KERNELSOURCEPATH}{\$HOME/kernel}

%Format
\newcommand{\interline}{5mm}

%Unit
\newcommand{\bits}{bits}
\newcommand{\bitsshort}{b}
\newcommand{\bytes}{bytes}
\newcommand{\bytesshort}{B}
\newcommand{\kibi}{Kibi}
\newcommand{\kibishort}{Ki}
\newcommand{\mebi}{Mebi}
\newcommand{\mebishort}{Mi}
\newcommand{\gibi}{Gibi}
\newcommand{\gibishort}{Gi}
\newcommand{\rpm}{RPM }
\newcommand{\msymbol}{' }
\newcommand{\s}{s }
\newcommand{\ssymbol}{'' }
\newcommand{\segundos}{segundos }
\newcommand{\segundo}{segundo }
\newcommand{\ms}{ms }
\newcommand{\milisegundos}{milisegundos }


\author{Introducción a los Sistemas Operativos}

\begin{document}

\title{\textit{Talk Show}}
\subtitle{Práctca 1}
\begin{frame}
  \titlepage
\end{frame}

\begin{frame}
  \begin{center}
    \vfill
    \huge ¿Qué es una partición?
    \vfill
  \end{center}
\end{frame}

\begin{frame}
  \frametitle{¿Qué es una partición?}
  \begin{itemize}
	  \item División lóica de un disco físico
	  \item Separación de datos
	  \item Organización de la información
	  \item ¿Ventajas?
	  \item ¿Desventajas?
  \end{itemize}
\end{frame}

\begin{frame}
  \begin{center}
    \vfill
    \huge ¿Qué tipos de particiones hay?
    \vfill
  \end{center}
\end{frame}

\begin{frame}
  \frametitle{¿Qué tipos de particiones hay?}
  \begin{itemize}
	  \item Primarias
	  \item Extendidas
	  \item Lógicas
	  \item ¿Cuántas se pueden tener?
	  \item ¿Cuáles tienen formato?
  \end{itemize}
\end{frame}

\begin{frame}
  \begin{center}
    \vfill
    \huge ¿Cómo se denominan los discos y las particiones en GNU/Linux?
    \vfill
  \end{center}
\end{frame}

\begin{frame}
  \frametitle{¿Cómo se denominan los discos y las particiones en GNU/Linux?}
  \begin{itemize}
	  \item Discos: sda, sdb, sdc, sdd
	  \item Particiones:
	  \begin{itemize}
	  	\item sda1, sda2, sda3, sda4, sdb1, ...
	  	\item sda5, sda6, ...
	  \end{itemize}
	  \item Según el FHS, ¿dónde se ubican los archivos que identi can los dispositivos?
  \end{itemize}
\end{frame}

\begin{frame}
  \begin{center}
    \vfill
    \huge ¿Cuántas particiones \textbf{se \underline{requieren} como mínimo} para instalar GNU/Linux?
    \vfill
  \end{center}
\end{frame}

\begin{frame}
  \begin{center}
    \vfill
    \huge /
    \vfill
  \end{center}
\end{frame}

\begin{frame}
  \begin{center}
    \vfill
    \huge ¿Cuántas particiones \textbf{se \underline{recomienda} tener como mínimo} para instalar GNU/Linux?
    \vfill
  \end{center}
\end{frame}

\begin{frame}
  \begin{center}
    \vfill
    \huge /
    \vfill
    \vfill
    \huge swap
    \vfill
  \end{center}
\end{frame}

\begin{frame}
  \begin{center}
    \vfill
    \huge Si quisieran instalar GNU/Linux para un \textbf{servidor de base de datos}.
    \vfill
    \vfill
    \huge Este tipo de servicio genera \textbf{gran cantidad de datos variables}.
    \vfill
    \vfill
    \huge ¿Cuántas particiones crearían? ¿cuáles y por qué?
    \vfill
  \end{center}
\end{frame}

\begin{frame}
  \begin{center}
    \vfill
    \huge /
    \vfill
    \vfill
    \huge swap
    \vfill
    \vfill
    \huge /var
    \vfill
  \end{center}
\end{frame}

\begin{frame}
  \begin{center}
    \vfill
    \huge ¿Qué es la \textbf{shell}?
    \vfill
  \end{center}
\end{frame}

\begin{frame}
  \frametitle{¿Qué es la shel?}
  \begin{itemize}
    \item Es un programa
    \item Interfaz entre el usuario y el Sistema Operativo
    \item Tiene comandos internos
    \item Invoca comandos externos \textbf{(\$PATH)}
    \item Con gurable por usuario:
    \begin{itemize}
      \item /etc/passwd
      \item chsh
    \end{itemize}

  \end{itemize}
\end{frame}

\begin{frame}
  \frametitle{¿Qué \textbf{comandos} debo ejecutar para hacer esto?}
  \begin{center}
     Se desea crear en el \textbf{/home} un directorio llamado \textbf{iso} y dentro de este dos más llamados \textbf{explicaciones} y \textbf{practicas}. Además se quiere mover el contenido de \textbf{/tmp/info-iso/explicaciones} a \textbf{/home/iso/explicaciones} y el de \textbf{/tmp/info-iso/practicas} a \textbf{/home/iso/practicas}
  \end{center}
\end{frame}

\end{document}