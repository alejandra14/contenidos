\begin{questions}

\question Características de \textit{GNU/Linux}:
\begin{parts}
  \part Mencione y explique las características más relevantes de \textit{GNU/Linux}.
  \part Mencione otros sistemas operativos y compárelos con \textit{GNU/Linux} en cuanto a los
  puntos mencionados en el inciso \textit{a}.
  \part ¿Qué es \textbf{GNU}?
  \part Indique una breve historia sobre la evolución del proyecto \textit{GNU}
  \part Explique qué es la multitarea, e indique si \textit{GNU/Linux} hace uso de ella.
  \part ¿Qué es \textbf{POSIX}?
\end{parts}

\question Distribuciones de \textit{GNU/Linux}:
\begin{parts}
  \part ¿Qué es una distribución de \textit{GNU/Linux}? Nombre al menos 4 distribuciones de
  \textit{GNU/Linux} y cite diferencias básicas entre ellas.
  \part ¿En qué se diferencia una distribución de otra?
  \part ¿Qué es \textbf{Debian}? Acceda al sitio \footnote{\url{https://www.debian.org/intro/about}} e indique cuáles son
  los objetivos del proyecto y una breve cronología del mismo
\end{parts}

\question Estructura de \textit{GNU/Linux}:
\begin{parts}
  \part Nombre cuales son los 3 componentes fundamentales de \textit{GNU/Linux}.
  \part Mencione y explique la estructura básica del Sistema Operativo \textit{GNU/Linux}.
\end{parts}

\question \textit{Kernel}:
\begin{parts}
  \part ¿Qué es? Indique una breve reseña histórica acerca de la evolución del Kernel de
  \textit{GNU/Linux}.
  \part ¿Cuáles son sus funciones principales?
  \part ¿Cuál es la versión actual? ¿Cómo se definía el esquema de versionado del Kernel en versiones anteriores a la 2.4? ¿Qué cambió en el versionado se impuso a partir de la versión 2.6?
  \part ¿Es posible tener más de un Kernel de \textit{GNU/Linux} instalado en la misma máquina?
  \part ¿Dónde se encuentra ubicado dentro del File System?
  \part ¿El Kernel de \textit{GNU/Linux} es monolítico? Justifique.
\end{parts}

\question Intérprete de comandos \textit{(Shell))}:
\begin{parts}
  \part ¿Qué es?
  \part ¿Cuáles son sus funciones?
  \part Mencione al menos 3 intérpretes de comandos que posee \textit{GNU/Linux} y compárelos
  entre ellos.
  \part ¿Dónde se ubican (\textit{path}) los comandos propios y externos al Shell?
  \part ¿Por qué considera que el Shell no es parte del Kernel de \textit{GNU/Linux}?
  \part ¿Es posible definir un intérprete de comandos distinto para cada usuario? ¿Desde
  dónde se define? ¿Cualquier usuario puede realizar dicha tarea?
\end{parts}

\question Sistema de Archivos \textit{(File System)}:
\begin{parts}
  \part ¿Qué es?
  \part Mencione sistemas de archivos soportados por \textit{GNU/Linux}.
  \part ¿Es posible visualizar particiones del tipo \textbf{FAT} y \textbf{NTFS} en \textit{GNU/Linux}?
  \part ¿Cuál es la estructura básica de los File System en \textit{GNU/Linux}? Mencione los
  directorios más importantes e indique qué tipo de información se encuentra en ellos.
  ¿A qué hace referencia la sigla \textbf{FHS}?
\end{parts}

\question Particiones:
\begin{parts}
  \part Definición. Tipos de particiones. Ventajas y Desventajas.
  \part ¿Cómo se identifican las particiones en \textit{GNU/Linux}? (Considere discos \textbf{IDE}, \textbf{SCSI} y
  \textbf{SATA}).
  \part ¿Cuántas particiones son necesarias como mínimo para instalar \textit{GNU/Linux}?
  Nómbrelas indicando tipo de partición, identificación, tipo de File System y punto de
  montaje.
  \part Ejemplifique diversos casos de particionamiento dependiendo del tipo de tarea que
  se deba realizar en su sistema operativo.
  \part ¿Qué tipo de software para particionar existe? Menciónelos y compare.
\end{parts}

\question Arranque (\textit{bootstrap}) de un Sistema Operativo:
\begin{parts}
  \part ¿Qué es el \textbf{BIOS}? ¿Qué tarea realiza?
  \part ¿Qué es \textbf{UEFI}? ¿Cuál es su función?
  \part ¿Qué es el \textbf{MBR}? ¿Que es el \textbf{MBC}?
  \part ¿A qué hacen referencia las siglas \textbf{GPT}? ¿Qué sustituye? Indique cuál es su
  formato.
  \part ¿Cuál es la funcionalidad de un “Gestor de Arranque”? ¿Qué tipos existen? ¿Dónde
  se instalan? Cite gestores de arranque conocidos.
  \part ¿Cuáles son los pasos que se suceden desde que se prende una computadora
  hasta que el Sistema Operativo es cargado (proceso de \textit{bootstrap})?
  \part Analice el proceso de arranque en \textit{GNU/Linux} y describa sus principales pasos.
  \part ¿Cuáles son los pasos que se suceden en el proceso de parada (\textit{shutdown}) de
  \textit{GNU/Linux}?
  \part ¿Es posible tener en una PC \textit{GNU/Linux} y otro Sistema Operativo instalado?
  Justifique.
\end{parts}

\question Archivos:
\begin{parts}
  \part ¿Cómo se identifican los archivos en \textit{GNU/Linux}?
  \part Investigue el funcionamiento de los editores \textbf{vi} y \textbf{mcedit}, y los comandos \textbf{cat} y
  \textbf{more}.
  \part Cree un archivo llamado “prueba.exe” en su directorio personal usando el vi. El
  mismo debe contener su número de alumno y su nombre.
  \part Investigue el funcionamiento del comando file. Pruébelo con diferentes archivos.
  ¿Qué diferencia nota?
\end{parts}

\question Indique qué comando es necesario utilizar para realizar cada una de las siguientes acciones. Investigue su funcionamiento y parámetros más importantes:
\begin{parts}
  \part Cree la carpeta ISO2017
  \part Acceda a la carpeta (cd)
  \part Cree dos archivos con los nombres iso2017-1 e iso2017-2  (touch)
  \part Liste el contenido del directorio actual (ls)
  \part Visualizar la ruta donde estoy situado  (pwd)   
  \part Busque todos los archivos en los que su nombre contiene la cadena “iso*” (find)
  \part Informar la cantidad de espacio libre en disco  (df)
  \part Verifique los usuarios conectado al sistema (who)
  \part Acceder a el archivo iso2017-1 e ingresar Nombre y Apellido
  \part Mostrar en pantalla las últimas líneas de un archivo (tail). 
\end{parts}

\question Investigue su funcionamiento y parámetros más importantes:
\begin{parts}
  \part shutdown
  \part reboot 
  \part halt    
  \part locate                                                 
  \part uname   
  \part gmesg                                               
  \part lspci
  \part at
  \part netstat 
  \part mount  
  \part umount
  \part head 
  \part losetup                                               
  \part write   
  \part mkfs                                                 
  \part fdisk (con cuidado)
\end{parts}

\question Investigue su funcionamiento y parámetros más importantes:
\begin{parts}
    \part Indique en qué directorios se almacenan los comandos mencionados en el ejercicio anterior.
\end{questions}
