\begin{questions}
\question Explique a que hacen referencia los siguientes términos:
\begin{itemize}
	\item Dirección Lógica o Virtual
	\item Dirección Física
\end{itemize}

\question En la técnica de Particiones Múltiples, la memoria es divida en varias particiones y los procesos son ubicados en estas, siempre que el tamaño del mismo sea menor o igual que el tamaño de la partición. Al trabajar con particiones se pueden considerar 2 métodos (independientes entre si):
\begin{itemize}
	\item Particiones Fijas
	\item Particiones Dinámicas
\end{itemize}
\begin{parts}
	\part Explique como trabajan estos 2 métodos. Cite diferencias, ventajas y desventajas.
	\part ¿Qué información debe disponer el SO para poder administrar la memoria con estos métodos?
	\part Realice un gráfico indicado como realiza el SO la transformación de direcciones lógicas a direcciones físicas.
\end{parts}

\question Al trabajar con particiones fijas, los tamaños de las mismas se pueden considerar:
\begin{itemize}
	\item Particiones de igual tamaño.
	\item Particiones de diferente tamaño.
\end{itemize}
Cite ventajas y desventajas de estos dos métodos.

\question Fragmentación.

Ambos métodos de particiones presentan el problema de la fragmentación:
\begin{itemize}
	\item Fragmentación Interna (Para el caso de Particiones Fijas)
	\item Fragmentación Externa (Para el caso de Particiones Dinámicas)
\end{itemize}
\begin{parts}
	\item Explique a que hacen referencia estos 2 problemas
	\item El problema de la Fragmentación Externa es posible de subsanar. Explique una técnica que evite este problema.
\end{parts}

\question Paginación
\begin{parts}
	\part Explique como trabaja este método de asignación de memoria.
	\part¿Qué estructuras adicionales debe poseer el SO para llevar a cabo su implementación?
	\part Explique, utilizando gráficos, como son transformadas las direcciones lógicas en físicas.
	\part En este esquema: ¿Se puede producir fragmentación (interna y/o externa)?
\end{parts}

\question Cite similitudes y diferencias entre la técnica de paginación y la de particiones fijas.

\question Suponga un sistema donde la memoria es administrada mediante la técnica de paginación, y donde:
\begin{itemize}
	\item El tamaño de la página es de 512 \bytes.
	\item Cada dirección de memoria referencia 1 \byte.
	\item Los marcos en memoria principal de encuentran desde la dirección física 0.
\end{itemize}	

Suponga además un proceso con un tamaño 2000 \bytes y con la siguiente tabla de páginas:
\begin{table}[h]
  \centering
  \resizebox{12pc}{!}{
  \begin{tabular}{| c | c |}
      \hline
      \bf Página & \bf Marco \\
      \hline
      0 & 3 \\
      \hline
      1 & 5 \\
      \hline
      2 & 2 \\
      \hline
      3 & 6 \\      
      \hline
  \end{tabular}
  }
\end{table}
\begin{parts}
	\part Realice los gráficos necesarios (de la memoria, proceso y tabla de paginas) en el que reflejen el estado descrito.
	\part Indicar si las siguientes direcciones lógicas son correctas y en caso afirmativo indicar la dirección física a la que corresponden:
	\begin{subparts}
		\subpart 35
		\subpart 512
		\subpart 2051
		\subpart 0
		\subpart 1325
		\subpart 602
	\end{subparts}
	\part Indicar, en caso de ser posible, las direcciones lógicas del proceso que se corresponden si las siguientes direcciones físicas:
		\begin{subparts}
		\subpart 509
		\subpart 1500
		\subpart 0
		\subpart 3215
		\subpart 1024
		\subpart 2000
	\end{subparts}
	\part ¿Indique, en caso que se produzca, la fragmentación (interna y/o externa)?
\end{parts}

\question Considere un espacio lógico de 8 paginas de 1024 bytes cada una, mapeadas en una memoria física de 32 marcos.
\begin{parts}
	\part ¿Cuántos bits son necesarios para representar una dirección lógica?
	\part ¿Cuántos bits son necesarios para representar una dirección física?
\end{parts}

\question Segmentación.
\begin{parts}
	\part Explique como trabaja este método de asignación de memoria.
	\part ¿Qué estructuras adicionales debe poseer el SO para llevar a cabo su implementación?
	\part Explique, utilizando gráficos, como son transformadas las direcciones lógicas en físicas.
	\part En este esquema: ¿Se puede producir fragmentación (interna y/o externa)?
\end{parts}

\question Cite similitudes y diferencias entre la técnica de segmentación y la de particiones dinámicas.

\question Cite similitudes y diferencias entre la técnica de paginación y segmentación.

\end{questions}
