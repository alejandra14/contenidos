\section{Conceptos teóricos}
\textit{El propósito de esta primera sección de la práctica es introducir
  los conceptos preliminares que necesitará el alumno, para desarrollar la
  actividad práctica del apartado B de la presente guía de estudio.}
\begin{questions}
  \question ¿Qué es el \textit{kernel} de GNU/Linux? ¿Cuáles son sus funciones
  principales dentro del Sistema Operativo?

  \question Indique una breve reseña histórica acerca de la evolución del
  \textit{kernel} de GNU/Linux

  \question Explique brevemente la arquitectura del \textit{kernel} de GNU/Linux
  teniendo en cuenta: tipo de \textit{kernel}, módulos, portabilidad, etc.

  \question ¿Cuáles son los cambios que se introdujeron en el \textit{kernel} a
  partir de la versión 3.0? ¿ Cuál fue la razón por la cual se cambió de la
  versión 2 a la 3? ¿Y la razón para el cambio de la versión 3 a la 4?

  \question ¿Cómo se define el versionado de los \textit{kernel}s de GNU/Linux?

  \question ¿Cuáles son las razones por las cuáles los usuarios de
  GNU/Linux recompilan sus kernels?

  \question ¿Cuáles son las distintas opciones para realizar la
  configuración de opciones de compilación de un \textit{kernel}? Cite diferencias,
  necesidades (paquetes adicionales de software que se pueden requerir),
  pro y contras de cada una de ellas.

  \question Nombre al menos 5 opciones de las más importantes que
  encontrará al momento de realizar la configuración de un \textit{kernel} para su
  posterior compilación.

  \question Indique que tarea realiza cada uno de los siguientes comandos
  durante la tarea de configuración/compilación del \textit{kernel}:

  \begin{parts}
  \part \texttt{make menuconfig}

  \part \texttt{make clean}

  \part \texttt{make} (investigue la funcionalidad del parámetro \texttt{-j})

  \part \texttt{make modules} (utilizado en antiguos kernels, actualmente
  no es necesario)

  \part \texttt{make modules\_install}

  \part \texttt{make install}
  \end{parts}

  \question Una vez que el \textit{kernel} fue compilado, ¿dónde queda ubicada su
  imagen? ¿dónde debería ser reubicada? ¿Existe algún comando que realice
  esta copia en forma automática?

  \question ¿A qué hace referencia el archivo \texttt{initramfs}? ¿Cuál es
  su funcionalidad?  ¿Bajo qué condiciones puede no ser necesario?

  \question ¿Cuál es la razón por la que una vez compilado el nuevo \textit{kernel},
  es necesario reconfigurar el gestor de arranque que tengamos instalado?

  \question ¿Qué es un módulo del \textit{kernel}? ¿Cuáles son los comandos
  principales para el manejo de módulos del kernel?

  \question ¿Qué es un parche del \textit{kernel}? ¿Cuáles son las razones
  principales por las cuáles se deberían aplicar parches en el kernel? ¿A
  través de qué comando se realiza la aplicación de parches en el
  \textit{kernel}?
\end{questions}

\section{Ejercicio taller: Compilación del \textit{kernel} Linux}
\textit{El propósito del siguiente ejercicio es el de guiar al alumno en el
  proceso de compilación del \textit{kernel} de GNU/Linux. Si bien los
  siguientes ejercicios permiten agregar determinada funcionalidad al
  \textit{kernel}, es aconsejable que los alumnos investiguen las distintas
  opciones con el fin de adquirir experiencia adicional. Para la
  realización de este taller se ha utilizado la versión \KERNELBASEVERSION\
  del \textit{kernel} de GNU/Linux. Aquel alumno que decida utilizar otra
  versión deberá descargar el código fuente y los parches del
  \textit{kernel} correspondientes de acuerdo a la versión utilizada.  }

A través de los siguientes pasos agregaremos nueva funcionalidad a nuestro
\textit{kernel} de GNU/Linux asumiendo que la misma no se encuentra soportada en
nuestro \textit{kernel} actual. Dentro de la funcionalidad a agregar se encuentran:

\begin{itemize}
\item El soporte para sistemas de archivos \texttt{minix} y \texttt{ext4}
\item Soporte para la utilización de dispositivos de \textit{loopback}
\item Actualizaremos una versión de nuestro \textit{kernel}, a través de la aplicación de
parches
\end{itemize}

Aquel alumno que utilice la máquina virtual provista por la Cátedra, no
tendrá que instalar el software necesario para realizar la compilación, ya
que el mismo se encuentra incluido. \textbf{El nombre del usuario a utilizar es
\texttt{so}, y su contraseña \texttt{sistemasoperativos}}. Este usuario
tiene permitido utilizar \texttt{sudo}, por lo cual para obtener
privilegios de \texttt{root} basta con ejecutar el comando \texttt{sudo
  su}.

En caso de no utilizar la máquina virtual provista, se deberá realizar la
instalación del software requerido para la instalación (librerías,
compiladores, etc.)

En los comandos de ejemplo de esta práctica se verá que algunos comandos
empiezan con \texttt{\$} y otros con \texttt{\#}. Estos símbolos
representan el \textit{prompt} del usuario y no deben escribirse cuando se
copie el comando. El símbolo \texttt{\$} significa que el comando debe
ejecutarse con un usuario normal. En el caso de la máquina virtual, es el
usuario \texttt{so}. El símbolo \texttt{\#} significa que el comando debe
ejecutarse con privilegios de usuario \texttt{root}.

\begin{questions}
  \question Descargue los archivos publicados en el sitio web de la
  cátedra, que se llaman \textit{minixFS} y \textit{ext4FS.gz} (en la
  máquina virtual provista por la cátedra se encuentran en el directorio
  \texttt{/home/so}).

  Estos archivos representan lo que se conoce como \textit{loop
    device}. Básicamente son archivos regulares que pueden ser tratados
  como dispositivos de bloques, donde uno puede crear un sistema de
  archivos dentro de ellos y montarlos como si montáramos cualquier otro
  dispositivo de bloques como, por ejemplo, particiones de un disco
  rígido. La particularidad que tienen estos archivos, es que han sido
  formateados con distintos tipos de sistemas de archivos. El archivo
  denominado \textit{minixFS} se formateó con un tipo de sistema de
  archivos denominado \texttt{minix}. Como ya sabrán, para poder acceder a
  la información de este pseudo dispositivo, tendremos que soportar el
  sistema de archivos \texttt{minix} en nuestro \textit{kernel}. Por otro lado, el
  archivo llamado \textit{ext4FS.gz} está comprimido con la utilidad
  \texttt{gzip} (para descomprimirlo utilizaremos \texttt{gzip -d
    ext4FS.gz}). En este archivo comprimido, encontraremos otro dispositivo
  de bloques, formateado con el sistema de archivos \textit{extended 4}
  (\texttt{ext4}).

  \question La tarea que tendremos que realizar en este ejercicio consiste
  en intentar montar los archivos con formato \texttt{minix} y
  \texttt{ext4} que se encuentran publicado en el sitio de la cátedra y que
  ya hemos descargado, con el fin de poder visualizar la información que en
  él se encuentra. Para ello, seguiremos los siguientes pasos:
  \begin{parts}
    \part Descargaremos este archivo en algún directorio determinado de
    nuestro \textit{File System}, como por ejemplo nuestro directorio
    personal, \texttt{\$HOME}.

    \part Verificaremos que dentro del directorio \texttt{/mnt} existan al
    menos dos directorios donde podamos montar nuestros pseudo
    dispositivos. Si no existen los directorios, los crearemos. Por ejemplo
    podemos utilizar el nombre \texttt{/mnt/minix/} y \texttt{/mnt/ext4/}.

    \part A continuación montaremos nuestros dispositivos utilizando los
    siguientes comandos:\\
    \texttt{\\
\$ sudo su \\
\# mount -t minix \$HOME/minixFS /mnt/minix/ -o loop\\
\# mount -t ext4 \$HOME/ext4FS /mnt/ext4/ -o loop}\\

\part ¿Puede explicar qué sucedió? {\small \textbf{Nota}: \textit{si estás
    usando la máquina virtual de la cátedra, el comportamiento esperado es
    que los comandos anteriores fallen.}}

  \end{parts}

  \question Descargue el código fuente del \textit{kernel} a compilar, el parche
  correspondiente a la versión a la que se quiere llevar el código fuente,
  arme la estructura de directorios y aplique el parche (en la máquina
  virtual provista por la cátedra se encuentran en el directorio
  \texttt{/usr/src}).
  \begin{parts}
    \part  Descargar el código fuente:\\
    \texttt{\\ % Debería hacerlo con verbatim o lslisting, pero no resuelve
               % las macros. Lo dejo así hasta que encuentre una forma más
               % prolija de hacerlo.
\$ cd /usr/src/\\
\$ sudo su\\
\# wget
https://kernel.org/pub/linux/kernel/v4.x/linux-\KERNELBASEVERSION.tar.xz}\\

\part Descargar el parche correspondiente a la versión a la que quiere
llevar su código fuente. Si se está usando la máquina virtual provista por la cátedra,
el parche también se encuentra en \texttt{/usr/src}:\\
\texttt{\\
\# wget https://kernel.org/pub/linux/kernel/v4.x/patch-\PATCHEDKERNELVERSION.xz} \\

\part Descomprimir el \textit{kernel} utilizando usuario sin privilegios de
\texttt{root} (en la virtual, el usuario \texttt{so}), en un nuevo
directorio en el \texttt{\$HOME} del usuario.\\
\texttt{\\
\# exit \\
\$ mkdir \$HOME/kernel/ \\
\$ cd \$HOME/kernel/ \\
\$ tar xvf /usr/src/linux-\KERNELBASEVERSION.tar.xz} \\

\part  Aplique el parche descargado a nuestro código fuente. Esta
tarea la haremos ubicados en el directorio de nuestro código fuente y a
través de la herramienta \texttt{patch}, verificando que no arroje errores al
finalizar.\\
\texttt{\\
\$ cd \$HOME/kernel/linux-\KERNELBASEVERSION \\
\$ xzcat /usr/src/patch-\PATCHEDKERNELVERSION.xz | patch -p1} \\
  \end{parts}

  \question Configurar el código fuente del \textit{kernel} para su
  compilación. Esta tarea se realiza a través de los comandos vistos en el
  ejercicio 8 de la primera parte de esta práctica. En primer lugar, se
  deberá partir de la configuración del \textit{kernel} que está corriendo
  actualmente. Por convención, esta configuración se encuentra ubicada en
  \texttt{/boot}. La copiaremos al directorio donde se encuentra el \textit{kernel}
  parcheado.\\
\texttt{\\
\$ cp /boot/config-\$(uname -r) \$HOME/kernel/linux-\KERNELBASEVERSION/.config} \\

  En este caso utilizaremos la herramienta \texttt{menuconfig}. Para ello
  ejecutaremos:\\
\texttt{\\
\$ cd \$HOME/kernel/linux-\KERNELBASEVERSION \\
\$ make menuconfig} \\

  Este comando pondrá en pantalla un menú
  de configuración del código fuente del \textit{kernel} donde podremos
  seleccionar diversas opciones dependiendo de nuestro hardware y
  necesidades. Dentro de las opciones que debemos seleccionar con el fin de
  cumplir nuestro objetivo se encuentran:
  \begin{parts}
    \part Soporte para sistemas de archivos \texttt{minix}: En la opción
    \textit{File Systems} $\to$ \textit{Miscellaneous filesystems},
    tendremos que seleccionar \textit{Minix fs support}.

    \part Soporte para sistemas de archivos \texttt{ext4}: En la opción
    \textit{File Systems}, tendremos que seleccionar \textit{The Extended 4
      (ext4) filesystem}.

    \part Soporte para dispositivos de \textit{Loopback}: En la opción
    \textit{Device Drivers} $\to$ \textit{Block Devices}, tendremos que
    seleccionar \textit{Loopback device support}.
  \end{parts}

  La forma de movernos a través de este menú es utilizando las flechas del
  cursor, la tecla \textit{Enter} y la barra espaciadora. A través de esta
  última podremos decidir sobre determinada opción si la misma será
  incluida en nuestro \textit{kernel}, si será soportada a través de módulos, o bien
  si no se dará soporte a la funcionalidad (\texttt{<*>}, \texttt{<M>},
  \texttt{< >} respectivamente). Una vez seleccionadas las opciones
  necesarias, saldremos de este menú de configuración a través de la opción
  \texttt{Exit}, guardando los cambios.

  \question Luego de configurar nuestro \textit{kernel}, nos dispondremos a realizar
  la compilación del mismo y sus módulos.
    
  Para realizar la compilación deberemos ejecutar:

  \texttt{\$ make -j\textbf{X}}

  \textbf{X} deberá reemplazarse por la cantidad de procesadores con los
  que cuente su máquina. En máquinas con más de un procesador o núcleo, la
  utilización de este parámetro puede acelerar mucho el proceso de
  compilación, ya que ejecuta \textbf{X} \textit{jobs} o procesos para la
  tarea de compilación en forma simultánea. Este comando debemos ejecutarlo
  ubicados dentro del directorio donde se encuentra el código fuente del
  \textit{kernel} descargado
  (\texttt{\KERNELSOURCEPATH/linux-\KERNELBASEVERSION}). La ejecución de
  este último puede durar varios minutos, o incluso horas, dependiendo del
  tipo de hardware que tengamos en nuestra PC y las opciones que hayamos
  seleccionado al momento de la configuración. Una vez finalizado este
  proceso, debemos verificar que no haya arrojado errores. En caso de que
  esto ocurra debemos verificar de qué tipo de error se trata y volver a la
  configuración de nuestro \textit{kernel} para corregir los problemas. Una
  vez corregidos tendremos que volver a compilar nuestro
  \textit{kernel}. Previo a esta nueva compilación debemos correr el
  comando \texttt{make clean} para eliminar pasos inconclusos de
  compilaciones anteriores.

  \question Finalizado este proceso, debemos reubicar las nuevas imágenes
  en los directorios correspondientes, instalar los módulos, crear una
  imagen \texttt{initramfs} y reconfigurar nuestro gestor de arranque.
  \begin{parts}
    \part En primer lugar, para realizar la instalación de los módulos
    ejecutaremos:

    \texttt{\$ sudo make modules\_install}

    Recordar que debemos estar ubicados en el directorio
    \texttt{\KERNELSOURCEPATH/linux-\KERNELBASEVERSION}. Este comando
    copiará los módulos compilados al directorio
    \texttt{/lib/modules/\PATCHEDKERNELVERSION}.

    \part Para reubicar la imagen del \textit{kernel} disponemos de al
    menos dos métodos.
    \begin{enumerate}
    \item Utilizar el siguiente comando:\\
      \texttt{\\
        \$ sudo make install} \\
 
      Este comando reubicará todos los archivos creados durante el proceso
      de compilación. Este es el método \textbf{recomendado} puesto que,
      entre otros detalles, no sobre-escribirá ningún archivo en caso de
      que el \textit{kernel} que estemos instalando tenga la misma versión
      que el que está corriendo.

    \item Este método consiste en hacer lo mismo que el anterior, pero de
      forma manual. Se deben realizar la copia manual de cada uno de los
      archivos creados. Para ello debemos ejecutar los siguientes comandos:\\
      \texttt{\\
        \$ sudo su \\
        \# cp -i \KERNELSOURCEPATH/linux-\KERNELBASEVERSION/arch/i386/boot/bzImage /boot/vmlinuz-\PATCHEDKERNELVERSION \\
        \# cp -i \KERNELSOURCEPATH/linux-\KERNELBASEVERSION/System.map /boot/System.map-\PATCHEDKERNELVERSION \\
        \# cp -i \KERNELSOURCEPATH/linux-\KERNELBASEVERSION/.config
        /boot/config-\PATCHEDKERNELVERSION
      } \\

El primer comando copiará la imagen del \textit{kernel}, el segundo copiará el
archivo de mapa de símbolos correspondiente a la nueva imagen del
\textit{kernel}. El último comando nos permitirá mantener una copia del archivo de
configuración con el cual fue compilado nuestro \textit{kernel}. Si bien no es
necesario mantener este último archivo, es recomendable hacerlo, con el fin
de mantener el orden y una documentación de las opciones de compilación
utilizadas en la imagen compilada.
    \end{enumerate}

    \part Si en el inciso anterior se utilizó la opción de \texttt{make
      install}, entonces podremos saltear este paso. De lo contrario, para
    la creación de la imagen \texttt{initramfs} utilizaremos el
    comando: \\
    \texttt{\\
      \# mkinitramfs -o /boot/initrd.img-\PATCHEDKERNELVERSION\ \PATCHEDKERNELVERSION}\\
    
\part El último paso previo a realizar la prueba de nuestra imagen del
\textit{kernel}, es reconfigurar nuestro gestor arranque de manera tal que podamos
\textit{bootear} con la nueva imagen. Suponiendo que el gestor de arranque
que tenemos instalado es GRUB en su versión 2, ejecutaremos el comando:\\
\texttt{\\
\# update-grub2}\\

Este comando se encargará de agregar las entradas correspondientes en el
archivo de configuración \texttt{/boot/grub/grub.cfg}. Nuevamente, si
utilizamos \texttt{make install} para instalar el \textit{kernel}, entonces
es posible saltear este paso.

  \end{parts}

  \question Como último paso, a través del comando \texttt{reboot},
  reiniciaremos nuestro equipo y probaremos el nuevo \textit{kernel}
  recientemente compilado. Al momento de cargarse nuestro gestor de
  arranque, veremos una nueva entrada que hace referencia al nuevo
  \textit{kernel}. Para \textit{bootear}, seleccionamos esta entrada y
  verificamos que el sistema funcione correctamente. En caso de que el
  sistema no arranque con el nuevo \textit{kernel}, tendremos la opción de
  volver a \textit{bootear} con nuestro \textit{kernel} anterior para
  corregir los errores y realizar una nueva compilación.

  \question Repita el procedimiento del inciso dos. ¿Qué sucede ahora?

  Como nuestro \textit{kernel} tiene soporte para este tipo de \textit{File
    System}, entonces, ingresando a \texttt{/mnt/minix}, encontraremos la
  información contenida en este pseudo dispositivo. Nos encontraremos con
  una situación similar al querer montar el sistema de archivos
  \texttt{ext4}.

\end{questions}
