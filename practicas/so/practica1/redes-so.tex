\section{Redes y Sistemas Operativos}

\textit{En esta sección se aprenderá cómo integrar conceptos del sistema
  operativo, como redirecciones y \textbf{pipes}, con una red TCP/IP. Para
  ello se utilizará principalmente la herramienta
  \texttt{nc}. Investigue su funcionamiento y responda. \textbf{Tip:} \texttt{man nc}}

\begin{questions}
  \question Desarrolle un \textit{script} que permita al usuario chatear
  con otra instancia del \textbf{mismo} \textit{script}. Para ello, el
  \textit{script} deberá recibir como parámetro si va a funcionar como
  \textbf{(c)}liente o como \textbf{(s)}ervidor. También deberá recibir
  como parámetro un \textit{nickname} para el usuario. Por ejemplo, para
  invocar el \textit{script} en modo servidor con el \textit{nick}
  \textbf{jvg}, debería ejecutar:

  \begin{lstlisting}

 ./nc-chat.sh s jvg
\end{lstlisting}

Además, los mensajes transmitidos deben comenzar con la fecha, hora y
\textit{nick} del emisor. Por ejemplo:
\begin{lstlisting}
  
Thu Mar 12 13:03:14 ART 2015, jvg says:
Knock knock Neo.
\end{lstlisting}  

\end{questions}
