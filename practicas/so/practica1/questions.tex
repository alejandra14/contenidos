\section{Repaso general}

\textit{El objetivo de esta primera parte de la práctica es repasar los
  conceptos de \textbf{shell scripting} aprendidos en la materia
  \textbf{Introducción a los Sistemas Operativos}. Se realizará un repaso
  general sobre comandos los comandos más comunes y su uso en
  \textbf{scripts}}.

\begin{questions}
\question Hello world
\question Good bye cruel world!
\end{questions}

\section{Redes y Sistemas Operativos}

\textit{En esta sección se aprenderá cómo integrar conceptos del sistema
  operativo, como redirecciones y \textbf{pipes}, con una red TCP/IP.}

\begin{questions}
\question Hello world
\question Good bye cruel world!
\end{questions}

\section{Ejercicio a entregar}
Esta sección está abocada al desarrollo de una shell, que se continuará en prácticas siguientes. La misma deberá ser capaz de interpretar un conjunto de funcionalidades mucho más limitado que una de las implementaciones existentes como bash, etc.
Con tal fin deberá investigar y responder las siguientes preguntas:

\begin{enumerate}
\item \question ¿Qué es la shell? y para que sirve?

\item \question ¿Qué implementaciones de la shell existen?

\item \question ¿En qué espacio (usuario/kernel) la ubicaría?

\item \question Cuándo abre una terminal y tipea comandos, ¿qué shell es la encargada de interactuar?
 
\item Si pensamos en el funcionamiento de una shell básica podríamos detallarlos secuencialmente de la siguiente manera:
	\begin{itemize}
		   \item Esperar a que el usuario ingrese un comando

	              \item Parsear lo que el usuario tipeo para obtener el comando con sus respectivos argumentos

           	   \item Crear un nuevo proceso para ejecutar el comando ingresado por el usuario

	              \item En el nuevo proceso ejecutar el comando y retornar con el estado de dicha ejecución

           	   \item En el proceso padre, esperar a que el proceso hijo termine

		   \item Volver a empezar.
   	\end{itemize}

    	\begin{enumerate}
		\item Investigue las system call \textit{fork}:
		        \begin{enumerate}
				\item \question Que es lo que realiza?
				\item \question ¿Que retorna?
				\item \question ¿Para que podrian servir los valores que retorna?
				\item \question ¿Por que invocaria a la misma al implementar una shell?
	                    \end{enumerate}

		\item Investigue la system call \textit{exec}:
		          \begin{enumerate}
				\item \question ¿Para qué sirve?
				\item \question ¿Comó se comporta?
				\item \question ¿Cuáles son sus diferentes declaraciones POSIX?
	                     \end{enumerate}

		\item Investigue la system call \textit{wait}:
		        \begin{enumerate}
				\item \question ¿Para qué sirve?
				\item \question Sin ella, ¿qué sucedería, pensando en la implementación de la shell?
	                   \end{enumerate}
            \end{enumerate}

\item Ejercicio de \textit{shell} 1: implemente una shell capaz de 

Aclaraciones: 

\end{enumerate}
