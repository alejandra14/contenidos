\section{Repaso general}

\textit{El objetivo de esta primera parte de la práctica es repasar los
  conceptos de \textbf{shell scripting} aprendidos en la materia
  \textbf{Introducción a los Sistemas Operativos}. Se realizará un repaso
  general sobre comandos los comandos más comunes y su uso en
  \textbf{scripts}}.

\subsubsection{Preguntas de repaso}

\begin{questions}

\question ¿Qué es la shell? ¿Para qué sirve?

\question ¿En qué espacio (de usuario o de kernel) la ubicaría?

\question Si pensamos en el funcionamiento de una shell básica podríamos
  detallarlo secuencialmente de la siguiente manera:

  \begin{itemize}
    \item Esperar a que el usuario \textbf{ingrese} un comando
    \item \textbf{Procesar la entrada} del usuario y obtener el \textbf{comando} con sus parámetros
    \item Crear un \textbf{nuevo proceso} para ejecutar el comando, iniciarlo y esperar que retorne
    \item Presentar la \textbf{salida} (de haberla) al usuario
    \item Volver a empezar.
  \end{itemize}

  Este tipo de comportamiento, típico de las shell interactivas, se conoce como \textbf{REPL}
  (\textit{\textbf{R}ead-\textbf{E}val-\textbf{P}rint \textbf{L}oop}, Ciclo de leer, interpretar e imprimir).

  Analice cómo implementaría este ciclo básico de interpretación de \textit{scripts}.

\question Investigue las system call \texttt{fork}:
  \begin{parts}
    \part Que es lo que realiza?
    \part ¿Que retorna?
    \part ¿Para que podrian servir los valores que retorna?
    \part ¿Por que invocaria a la misma al implementar una shell?
  \end{parts}

\question Investigue la system call \texttt{exec}:
  \begin{parts}
    \part ¿Para qué sirve?
    \part ¿Comó se comporta?
    \part ¿Cuáles son sus diferentes declaraciones POSIX?
  \end{parts}

\question Investigue la system call \texttt{wait}:
  \begin{parts}
    \part ¿Para qué sirve?
    \part Sin ella, ¿qué sucedería, pensando en la implementación de la shell?
  \end{parts}

\end{questions}

\subsubsection{\textit{Scripts}}

\begin{questions}

\question Relice un \textit{script} que guarde en el archivo \texttt{/tmp/usuarios} los
  nombres de los usuarios del sistema cuyo \textit{UID} sea mayor a 1000.

\question Implemente un \textit{script} que reciba como parámetro el nombre de un
  proceso e informe cada 15 segundos cuántas instancias de ese proceso están en
  ejecución.

\question Desarrolle un \textit{script} que guarde en un arreglo todos los archivos del
  directorio actual (incluyendo sus subdirectorios) para los cuales el usuario que ejecuta
  el \textit{script} tiene permisos de \textbf{ejecución}. Luego, implemente las siguientes
  funciones:
  \begin{parts}
    \part \texttt{cantidad}: Imprime la cantidad de archivos que se encontraron
    \part \texttt{archivos}: Imprime los nombres de los archivos encontrados en orden alfabético
  \end{parts}

\question Se le ha encomendado organizar las fotos (en formato \texttt{jpg}) de todos los eventos
  de los que su empresa ha participado en el último año, los cuales se encuentran organizados en
  directorios con el nombre del evento. Para facilitar su búsqueda posterior, los archivos deben
  tener nombres que sigan el siguiente patrón: \texttt{EVENTO-N.jpg}, donde:
  \begin{itemize}
    \item \texttt{EVENTO} es el nombre del evento (el del directorio que se está procesando)
    \item \texttt{N} es un índice de foto, comenzando en \texttt{1}
  \end{itemize}

  Realice un \textit{script} que renombre los archivos de cada subdirectorio del directorio actual
  siguiendo lo especificado en el párrafo anterior.

  \textbf{Ejemplo:} dada la siguiente estructura de archivos y directorios:

  \begin{lstlisting}

bashconf-15/
  DSC01050.jpg
  DSC01051.jpg
  DSC01052.jpg
  DSC01053.jpg
  DSC01054.jpg
jsconf-14/
  DSC01230.jpg
  DSC01231.jpg
  DSC01232.jpg
  DSC01235.jpg
  DSC01236.jpg
oktoberfest-14/
  DSC02229.jpg
  DSC02230.jpg
  DSC02231.jpg
  DSC02232.jpg

  \end{lstlisting}

  Se desea terminar con la siguiente estructura luego de ejecutar su \textit{script}:

  \begin{lstlisting}

bashconf-15/
  bashconf-15-1.jpg
  bashconf-15-2.jpg
  bashconf-15-3.jpg
  bashconf-15-4.jpg
  bashconf-15-5.jpg
jsconf-14/
  jsconf-14-1.jpg
  jsconf-14-2.jpg
  jsconf-14-3.jpg
  jsconf-14-4.jpg
  jsconf-14-5.jpg
oktoberfest-14/
  oktoberfest-14-1.jpg
  oktoberfest-14-2.jpg
  oktoberfest-14-3.jpg
  oktoberfest-14-4.jpg

  \end{lstlisting}

\question Escriba un \textit{script} que liste en orden alfabético inverso el contenido
  del directorio actual. Es decir, si el contenido son los archivos:

  \begin{lstlisting}

archivo_1.txt articulo.doc directorio directorio_2 script.sh

  \end{lstlisting}

  se espera que el \textit{script} los liste de la siguiente manera:

  \begin{lstlisting}

script.sh directorio_2 directorio articulo.doc archivo_1.txt

  \end{lstlisting}

\question Realice un \textit{script} que \textbf{copie} todos los archivos del directorio home
  del usuario que lo ejecuta, a un subdirectorio del mismo llamado \texttt{backup} cambiándoles
  el nombre para que esté en \textbf{mayúsculas}. \textbf{No se deben procesar los subdirectorios
  del home del usuario}, únicamente los archivos ubicados directamente en este. Si el directorio
  \texttt{backup} existe al iniciar el script, el contenido del mismo debe borrarse antes de
  copiar los archivos.

  \textbf{Ejemplo:} si el home del usuario actual contiene:

  \begin{lstlisting}

/
  home/
    mi_usuario/
      so/
        practica1.pdf
        ejercicios/
          ejercicio-1.sh
          ejercicio-2.sh
      archivo1.txt
      mi-script.sh
      otro_archivo.txt

  \end{lstlisting}

se espera tener lo siguiente luego de la ejecución del \textit{script}:

\begin{lstlisting}

/
  home/
    mi_usuario/
      backup/
        ARCHIVO1.TXT
        MI-SCRIPT.SH
        OTRO_ARCHIVO.TXT
      so/
        practica1.pdf
        ejercicios/
          ejercicio-1.sh
          ejercicio-2.sh
      archivo1.txt
      mi-script.sh
      otro_archivo.txt

\end{lstlisting}

\question Un escritor tiene organizados los capítulos de su próximo libro en distintos archivos
  de texto plano en un mismo directorio, y le ha solicitado ayuda para \textbf{concatenar}
  el contenido de cada uno de ellos en un único archivo final llamado \texttt{libro.txt},
  de modo tal que éste último contenga el texto de todos los otros archivos, uno luego del otro.
  Puede asumir que los archivos de los capítulos tienen nombres alfabéticamente ordenados:
  \texttt{capitulo-01.txt}, \texttt{capitulo-02.txt}, ..., \texttt{capitulo-48.txt}, por ejemplo.

  \textbf{Tip:} \texttt{man cat}.

\end{questions}
