\section{Parte 1: Conceptos teóricos}

\begin{questions}
  \question Defina \textbf{virtualización}. Investigue cuál fue una de las
  primeras implementaciones que se realizó.

  \question ¿Qué diferencia existe entre \textbf{virtualización} y
  \textbf{emulación}?

  \question Investigue el concepto de \textit{hypervisor} y responda:
  \begin{parts}

    \part ¿Qué es un \textbf{\textit{hypervisor}}? Investigue cuándo se
    realizó la primer implementación de esta tecnología.

    \part ¿Qué beneficios traen los \textit{hypervisors}? ¿Cómo se
    clasifican?

    \part Indique por qué un \textit{hypervisor} de tipo 1 no podría
    correr en una arquitectura sin tecnología de virtualización. ¿Y un
    \textit{hypervisor} de tipo 2 en hardware sin tecnología de
    virtualización?
  \end{parts}

  \question Investigue el concepto de \textbf{paravirtualización} y
  responda:
  \begin{parts}
    \part ¿Qué es la \textbf{paravirtualización}?

    \part ¿Sería posible utilizar paravirtualización en sistemas operativos
    como Windows o iOS? ¿Por qué?

    \part Mencione algún sistema que implemente paravirtualización.

    \part Defina VMI.

    \part ¿Qué beneficios trae con respecto al resto de los modos de
    virtualización?

    \part Investigue si VMI podría correr sobre \textit{hypervisors} de
    tipo 1 o 2, y justifique por qué.
  \end{parts}

  \question ¿Qué es \textit{User-mode Linux}? ¿Qué diferencias encuentra
  con el resto de las tecnologías estudiadas?

  \question Investigue sobre \textit{\textbf{containers}} en el ámbito de
  la virtualización y responda:
  \begin{parts}
    \part ¿Qué son?
    \part ¿Dependen del \textit{hardware} subyacente?
    \part ¿Qué lo diferencia por sobre el resto de las tecnologías
    estudiadas?
    \part Investigue qué funcionalidades del \textit{kernel} Linux permiten
    la implementación de \textit{containers}.
  \end{parts}

\end{questions}

\section{Parte 2: \textit{Service isolation}}
\textit{Se plantea el escenario donde usted es un administrador de sistemas
  que requiere ejecutar un servicio de forma aislada al resto de los
  procesos.}

\subsection{Ejecución normal}
\begin{questions}
  \question Es la forma más sencilla de ejecutar un servicio y consiste
  simplemente en correrlo como cualquier otro proceso. Ejecute el siguiente
  \textit{script} que símplemente informa al administrador datos básicos
  sobre el sistema operativo, lista el contenido del directorio
  \texttt{/home} y muestra la cantidad de procesos corriendo.
  
  \lstinputlisting[language=sh]{script.sh}

  Este sencillo \textit{script} es capaz de acceder a información del
  sistema operativo como cualquier otro servicio nativo. Esta información
  consiste en archivos y directorios sobre los cuales se tengan permisos de
  lectura, procesos en ejecución, información sobre particiones, interfaces
  de red, etc. En algunos casos suele ser conveniente restringir la
  cantidad de información a la que un proceso puede acceder.
  
\end{questions}

\subsection{\texttt{chroot}}
Uno de los métodos más simples para aislar servicios es \texttt{chroot},
que consiste simplemente en cambiar lo que un proceso junto con sus hijos
consideran que es el directorio raíz, limitando de esta forma lo que pueden
ver en el sistema de archivos. En esta sección de la práctica se preparará
un árbol de directorios que sirva como directorio raíz para una nueva
\textit{shell}.

\begin{questions}
  \question ¿Cuál es la finalidad de la herramienta \texttt{debootstrap}?
  Instálela en un sistema operativo basado en Debian.

  \question La sintaxis de \texttt{debootstrap} para crear un sistema base
  es: \texttt{debootstrap\ <suite>\ <target>\ <mirror>}. Utilice como
  \textit{suite} la versión estable de Debian, \texttt{stable}, como
  \textit{target} el directorio donde alojará el árbol de directorios y
  como \textit{mirror} \texttt{http://httpredir.debian.org/debian}.

  \question Para que un \texttt{chroot} funcione correctamente en Linux se
  deben montar ciertos sistemas de archivos API que necesita el sistema
  operativo.
\begin{verbatim}
# mount --bind /dev/ target/dev/
# mount --bind /proc/ target/proc/
# mount --bind /sys/ target/sys/
\end{verbatim}

  Tenga en cuenta que debe reemplazar \texttt{target/} por el directorio
  donde inicializó su \textit{debootstrap}.

  \question Copie el \textit{script} que ejecutó en la sección anterior a
  un directorio accesible desde el \texttt{chroot}. Por ejemplo,
  \texttt{target/bin/script.sh}.

  \question Ejecute una \textit{shell} cuyo directorio raíz sea
  \texttt{target/}. Para tal fin, utilice con privilegios de \texttt{root}
  el comando \texttt{chroot}.

  \question Ejecute el \textit{script} instalado previamente y analice los
  resultados. ¿Puede ver los procesos que corren en el sistema operativo
  base? ¿Por qué?

  \question ¿Qué desventajas encuentra en el uso de \texttt{chroot} como
  medida de seguridad?

\end{questions}

\subsection{\textit{Containers}}

\textit{En GNU/Linux hay varias implementaciones de la tecnología de
  containers que provee el kernel. En esta sección de la práctica
  estudiaremos una de las primeras, \textbf{LXC (Linux Containers)}.}

Ejercicio guiado:
\begin{questions}

  \question Instale LXC mediante el comando:
\begin{verbatim}
# apt-get install lxc
\end{verbatim}
  
  \question Compruebe mediante el siguiente comando si su \textit{kernel} soporta
  LXC:
\begin{verbatim}
# lxc-checkconfig
\end{verbatim}
  \textbf{Nota}: puede que no todas las opciones estén habilitadas

  \question Compruebe los \textit{templates} que dispone (a partir de donde se
  crean los \textit{containers}):
\begin{verbatim}
# ls /usr/share/lxc/templates
\end{verbatim}
  
  \question Cree un nuevo \textit{container} desde el \textit{template} de Debian
  con el nombre \texttt{sodebian}:
\begin{verbatim}
# lxc-create -t debian -n sodebian
\end{verbatim}
  \textbf{Nota}: durante la instalación se muestra el usuario y la
  contraseña utilizados para ingresar al \textit{container}. Este comando
  puede demorar varios minutos la primera vez que es ejecutado ya que debe
  descargar muchos componentes.
  
  \question Una vez creado el \textit{container} inícielo mediante el siguiente
  comando:
\begin{verbatim}
# lxc-start -n sodebian -d
\end{verbatim}
  \textbf{Nota}: \texttt{-n} para ingresar el nombre del container que
  queremos iniciar, \texttt{-d} para que corra como \textit{daemon}.
  
  \question Utilice el comando \texttt{lxc-ls --fancy} para comprobar el estado
  de los \textit{containers}. ¿En qué estado se encuentra el
  \textit{container} \texttt{sodebian}?
  
  \question Mediante el siguiente comando ingrese al \textit{container} y
  ejecute el \textit{script} utilizado en los ejercicios anteriores.
\begin{verbatim}
# lxc-console -n sodebian
\end{verbatim}

  \begin{parts}
    \part ¿Puede acceder al \texttt{/home} del sistema operativo base?
    \part ¿Puede ver los procesos que están corriendo en el sistema
    operativo base?
  \end{parts}

  \question Detenga el \textit{container} \texttt{sodebian}.
\begin{verbatim}
# lxc-stop -n sodebian
\end{verbatim}
  
  \question Clone un \textit{container} utilizando el comando
  \texttt{lxc-clone}.

  \question Vuelva a ejecutar \texttt{lxc-ls} y analice su salida.

  \question Analice el contenido del directorio \texttt{/var/lib/lxc/sodebian/}.
  
  \question Elimine definitivamente el \textit{container} mediante el comando:
\begin{verbatim}
# lxc-destroy -n sodebian
\end{verbatim}  
\end{questions}

\subsection{Conclusión}
\begin{questions}
  \question Mediante los conceptos estudiados describa situaciones donde
  utilizaría:
  \begin{parts}
    \part Virtualización por \textit{hardware}.
    \part \textit{chroots}.
    \part \textit{Containers}.
  \end{parts}
\end{questions}

\section{Parte 3: \texttt{systemd}}
\texttt{systemd} es un reemplazo para el clásico inicio de los sistemas
operativos *nix, System V\footnote{¿recuerdan en ISO los \textit{scripts}
  en \texttt{/etc/init.d/}, los \textit{runlevels}, los enlaces en los
  directorios \texttt{/etc/rcX.d/}, etc.?}. Sin embargo, \texttt{systemd}
apunta a ser mucho más que solo un sistema de arranque de servicios, ya que
también provee funcionalidad para registrar eventos\footnote{mediante
  \texttt{\href{http://0pointer.de/blog/projects/journalctl.html}{journalctl}}},
invocar servicios cada cierto tiempo\footnote{reemplazando \texttt{cron} e
  incluso \texttt{anacron}}, arrancar servicios solo si se los
necesita\footnote{mediante una tecnología denominada
  \texttt{\href{http://0pointer.de/blog/projects/socket-activation.html}{socket
      activation}}}, entre otras cosas. Se recomienda la lectura del
artículo
\textit{\href{http://0pointer.de/blog/projects/systemd.html}{Rethinking PID
    1}} para profundizar más en \texttt{systemd}. En esta parte de la
práctica nos centraremos en aprender las capacidades para aislar
aplicaciones que \texttt{systemd} provee de forma \textit{built-in}.

\subsection{Ejecución normal}
En primer lugar crearemos un servicio similar al utilizado en la anterior
sección de la práctica. Aprovecharemos la capacidad que tiene
\texttt{systemd} de permitir que cada usuario defina sus propios
servicios. El alumno atento recordará que en System V todos los
\textit{scripts} de arranque debían definirse en \texttt{/etc/init.d/},
mientras que en los directorios \texttt{/etc/rcX.d/} se definía si
arrancaba o no en un determinado \textit{runlevel}.
\begin{questions}
  \question Para esta parte de la práctica se precisa un sistema operativo
  que corra \texttt{systemd} como sistema de arranque. Debian 8
  (\textit{Jessie}) y Ubuntu 15.04 (\textit{Vivid Vervet}) vienen por
  defecto con esta configuración. Elija la distribución de su preferencia e
  instalela en una máquina virtual o física.

  \question Utilizando un usuario que no sea \texttt{root}, cree el
  directorio donde se instalarán los servicios que serán utilizados durante
  el resto de la práctica.
\begin{verbatim}
$ mkdir -p $HOME/.config/systemd/user
\end{verbatim}

  \question Copie el \textit{script} utilizado anteriormente en, por
  ejemplo, \texttt{/usr/local/bin/script.sh}. Agréguele permisos de
  ejecución.

  \question Cree un
  \textit{\href{http://0pointer.de/public/systemd-man/systemd.unit.html}{Unit
      file}} para el servicio en la ubicación
  \texttt{\$HOME/.config/systemd/user/script.service} con el siguiente
  contenido: \lstinputlisting[language=sh]{script.service}

  \textbf{Nota:} reemplace la ruta de \texttt{ExecStart} por donde haya
  instalado el \textit{script}. % TODO: Hacer andar %h
  
  \question Ahora debemos decirle a \texttt{systemd} que refresque la lista
  de servicios del usuario:
\begin{verbatim}
$ systemctl --user daemon-reload
\end{verbatim}

    \question Finalmente, es hora de arrancar el servicio recién configurado:
\begin{verbatim}
$ systemctl --user start script.service
\end{verbatim}

    \question Para comprobar la salida estándar del \textit{script} recién
    ejecutado podemos utilizar \texttt{journalctl}, la herramienta que
    provee \texttt{systemd} para registrar eventos.
\begin{verbatim}
$ journalctl --user -f $HOME/bin/script.sh
\end{verbatim}

    Analice la salida del \textit{script}. ¿Corresponde a la información
    del sistema operativo base?   

\end{questions}

\subsection{\texttt{chroot}}
\texttt{systemd} permite integrar fácilmente a sus servicios la
funcionalidad que provee \texttt{chroot}.

\textbf{Nota:} debido a que se precisan montar los directorios API, esta
parte de la práctica la haremos utilizando el usuario \texttt{root}.
\begin{questions}
  \question Mediante \texttt{debootstrap} cree un nuevo \texttt{chroot} o
  utilice el de la sección anterior. Durante el resto del ejercicio se
  asumirá que lo tiene creado en \texttt{/home/chroot/sid/}.

  \question Copie el \textit{script} utilizado en la sección anterior a una
  ubicación dentro del \texttt{chroot}, por ejemplo:
  \texttt{/home/chroot/sid/usr/local/bin/script.sh}

  \question Cree el archivo
  \texttt{/etc/systemd/system/script-chroot.service} con el siguiente
  contenido:

  \lstinputlisting[language=sh]{script-chroot.service}

  \question Refresque la lista de servicios disponibles para
  \texttt{systemd}:
  \begin{verbatim}
$ systemctl daemon-reload
\end{verbatim}

    \question Finalmente, es hora de arrancar el servicio recién configurado:
\begin{verbatim}
$ systemctl start script-chroot.service
\end{verbatim}

    \question Para comprobar la salida estándar del \textit{script} recién
    ejecutado podemos utilizar \texttt{journalctl}, la herramienta que
    provee \texttt{systemd} para registrar eventos.
\begin{verbatim}
$ journalctl -u script-chroot.service
\end{verbatim}

    Analice la salida del \textit{script}. ¿Corresponde a la información
    del sistema operativo base?   

\end{questions}

\subsection{\textit{File system namespaces}}
En esta sección de la práctica dejaremos de lado los \texttt{chroot} para
dar lugar a una funcionalidad que provee el \textit{kernel} y que aprovecha
\texttt{systemd}: los \textit{file system namespaces}

\begin{questions}
  \question Investigue qué son los \textit{file system namespaces} que
  provee el kernel Linux.

  \question Instale el siguiente \textit{Unit file} en
  \texttt{\$HOME/.config/systemd/user/script-namespace.service}:

  \lstinputlisting[language=sh]{script-namespace.service}

  \question Investigue cuál es el efecto de utilizar la opción
  \texttt{InaccessibleDirectories}.

    \question Refresque la lista de servicios disponibles para
  \texttt{systemd}:
  \begin{verbatim}
$ systemctl --user daemon-reload
\end{verbatim}

    \question Finalmente, es hora de arrancar el servicio recién configurado:
\begin{verbatim}
$ systemctl --user start script-namespace.service
\end{verbatim}

    \question Para comprobar la salida estándar del \textit{script} recién
    ejecutado podemos utilizar \texttt{journalctl}, la herramienta que
    provee \texttt{systemd} para registrar eventos.
\begin{verbatim}
$ journalctl --user -u script-namespace.service
\end{verbatim}

    Analice la salida del \textit{script}. ¿Corresponde a la información
    del sistema operativo base?
\end{questions}

\subsection{\textit{Containers}}
Finalmente, \texttt{systemd} también provee funcionalidad \textit{built-in}
para ejecutar aplicaciones en \textit{containers} a través de su
herramienta \texttt{systemd-nspawn}. Si bien no dispone de tantos
\textit{features} como otras herramientas de \textit{containers},
\texttt{systemd-nspawn} es ideal para realizar \textit{testing},
\textit{debugging}, compilaciones en entornos limpios, etc.
\begin{questions}
  \question Mediante \texttt{debootstrap} cree un nuevo \texttt{chroot} o
  utilice el de la sección anterior. Durante el resto del ejercicio se
  asumirá que lo tiene creado en \texttt{/home/chroot/sid/}.

  \question Copie el \textit{script} utilizado en la sección anterior a una
  ubicación dentro del \texttt{chroot}, por ejemplo:
  \texttt{/home/chroot/sid/usr/local/bin/script.sh}

  \question Ejecute con privilegios de usuario \texttt{root} el siguiente
  comando:
\begin{verbatim}
systemd-nspawn -D /home/chroot/sid /usr/local/bin/script.sh
\end{verbatim}

  \question Analice la salida del \textit{script}. ¿Corresponde a la
  información del sistema operativo base?

  \question ¿Es posible ejecutar servicios de la forma que se hizo en las
  anteriores secciones pero a través de \texttt{systemd-nspawn}? En caso
  afirmativo, investigue cómo hacerlo.

\end{questions}