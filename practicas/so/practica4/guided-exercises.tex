\section{Ejercicios prácticos}

\subsection{Requisitos}
Para poder realizar los ejercicios prácticos debera contar con las siguientes herramientas:
\begin{itemize}
 \item El \textit{SDK} de Android. Descargue la última versión desde \href{https://developer.android.com/sdk/index.html\#Other}{esta}\footnote{https://developer.android.com/sdk/index.html\#Other} página:
 \begin{itemize}
      \item Desempaquete y descomprima el archivo. 
      
      \item Para facilitar el desarrollo de los ejercicios, agregue a algún directorio de la varaible de entorno \textit{PATH}, el \textit{path absoluto} de las siguientes herramientas:
      \begin{itemize}
	\item \textit{<android-sdk-dorectory>/platform-tools/adb}.
	\item \textit{<android-sdk-dorectory>/platform-tools/fastboot}.
	\item \textit{<android-sdk-dorectory>/platform-tools/sqlite3}.
	\item \textit{<android-sdk-dorectory>/tools/android}.
	\item \textit{<android-sdk-dorectory>/tools/emulator-x86}.
      \end{itemize}
      
      \item Acceda al \textit{Android SDK Manager}:
      \begin{lstlisting}
# android sdk
      \end{lstlisting}
     
      \item Descargue e instale/actualice los siguientes paquetes:
      \begin{itemize}
	\item \textit{Android SDK Tools}.
	\item \textit{Android SDK Platform-tools}.
	\item \textit{Android SDK Build-tools}.
	\item Android 4.2.2 (API 17) $\rightarrow$ \textit{SDK Platform}.
	\item Android 4.2.2 (API 17) $\rightarrow$ \textit{Intel x86 Atom System Image}.
      \end{itemize}
 \end{itemize}
 
 \item \textit{Linux image tools} para crear una imagen \textit{boot.img}. Descargue las mismas desde \href{http://dl.miniand.com/allwinnera10/system/image/tools.tar.gz}{esté}\footnote{http://dl.miniand.com/allwinnera10/system/image/tools.tar.gz} link:
 \begin{itemize}
      \item Desempaquete y descomprima el archivo.

      \item Para facilitar el desarrollo de los ejercicios, agregue a algún directorio de la varaible de entorno \textit{PATH}, el \textit{path absoluto} de la herramienta:
      \begin{itemize}
	\item \textit{<linux-image-tools-directory>/mkbootimg}.
      \end{itemize}
 \end{itemize}
 
 \item Herramienta para separar una imagen \textit{boot.img}. Descargue la última versión desde \href{http://whiteboard.ping.se/Android/Unmkbootimg}{esté}\footnote{http://whiteboard.ping.se/Android/Unmkbootimg} sitio:
 \begin{itemize}
      \item Descomprima el archivo.

      \item Para facilitar el desarrollo de los ejercicios, agregue a algún directorio de la varaible de entorno \textit{PATH}, el \textit{path absoluto} de la herramienta:
      \begin{itemize}
	\item \textit{<Unmkbootimg-directory>/unmkbootimg}.
      \end{itemize}
 \end{itemize}
 
 \item Una \textit{custom ROM} de Android. Descargue la del \textbf{Samsung Galaxy S III (Cricket)} desde \href{https://download.cyanogenmod.org/get/jenkins/57776/cm-10.2.1-d2cri.zip}{esté}\footnote{https://download.cyanogenmod.org/get/jenkins/57776/cm-10.2.1-d2cri.zip} link:
 \begin{itemize}
      \item Descomprima el archivo.
 \end{itemize} 
\end{itemize}


\subsection{Puesta a punto}
\begin{itemize}
    \item Cree una maquina virtual ingresando la letra \textit{n} cuando se le realice la pregunta ``Do you wish to create a custom hardware profile [no]'':
    \begin{lstlisting}
# android create avd --target 1 --name emulador-so
    \end{lstlisting}

    \item Inicie el emulador (\textit{avd - android virtual device}):
    \begin{lstlisting}
# emulador-x86 -avd emulador-so
    \end{lstlisting}

    \item Acceda a la shell del dispositivo:
    \begin{lstlisting}
# adb shell
    \end{lstlisting}
    Esto creara un cliente \textit{adb} e iniciara el servidor \textit{adb}, el cual se comunicará con el demonio \textit{adbd} del dispositivo ejecutandose en \textit{background}.
\end{itemize}

\subsection{SQLite}
Para su ayuda, ejecute:
\begin{lstlisting}
		      # sqlite3
		      sqlite> .help
\end{lstlisting}

\begin{itemize}
    \item Acceda a la configuración del dispositivo:
    \begin{lstlisting}
# sqlite3 /data/data/com.android.providers.settings/databases/settings.db
    \end{lstlisting}

    \item Liste todos los valores de la tabla \emph{system}:
    \begin{lstlisting}
sqlite> select * from system;
    \end{lstlisting}
    
    \begin{itemize}
	\item Se puede apreciar el volumen del tono de llamada, el de las notificaciones, etc.    
    \end{itemize}
    
    \begin{questions}
	\question ¿Qué pasa si abrimos la configuración del dispositivo desde la interface gráfica (\emph{Menu -> Settings -> Sound}) y modificamos el volumen del tono de llamada o de notificaciones?
	
	\question ¿Y qué pasa si ejecutamos el comando \textit{delete from system;}?, ¿Sigue funcionando el sistema operativo?, ¿Qué pasa con las configuraciones?
    \end{questions}
\end{itemize}

\subsection{Aplicaciones}
En Android, la forma de ejecutar aplicaciones es mediante la herramienta \textit{am}. La misma provee la funcionalidad necesaria para ejecutar activities, services, entre otras cosas.
Para su ayuda, ejecute:
\begin{lstlisting}
		      # am
\end{lstlisting}

\begin{itemize}
    \item Ejecute la actividad \textit{main} de la aplicación \emph{com.android.browser}:
    \begin{lstlisting}
# am start -a android.intent.action.MAIN -n com.android.browser/.BrowserActivity
    \end{lstlisting}
    \begin{itemize}
	\item ¿Cómo funciona?:
	\begin{itemize}
	    \item El primer parámetro inica que se va a hacer. En este caso, se ejecuta una aplicación (\emph{start}).
	    
	    \item El segundo parámetro inica el tipo de acción que se ejecutará (\emph{android.intent.action.MAIN}). \textbf{Nota}: para ver más tipos de acciones visite \href{http://developer.android.com/guide/topics/intents/intents-filters.html}{está}\footnote{http://developer.android.com/guide/topics/intents/intents-filters.html} página.
	    
	    \item El tercer parámetro es la actividad que se desea mostrar (\emph{com.android.browser/.BrowserActivity}).
	\end{itemize}
    \end{itemize}

    \item Inicie la aplicación \emph{settings} del dispositivo:
    \begin{lstlisting}
# am start -a android.intent.action.MAIN -n com.android.settings/.Settings
    \end{lstlisting}    
\end{itemize}