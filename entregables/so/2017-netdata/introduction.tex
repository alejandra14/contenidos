Este trabajo consiste en modificar la herramienta \textit{netdata} para
graficar
información de la interfaz de red \textit{eth0} utilizando una system call
personalizada que copie la información necesaria a espacio de usuario.
La system call deberá acceder a las estructuras de datos internas del kernel
para obtener la información.

Se pide graficar:
\begin{itemize}
	\item Paquetes recibidos y transmitidos,
	\item Errores.
	\item Paquetes descartados (dropped).
	\item Multicasts.
	\item Colisiones.
\end{itemize}

La herramienta
\textit{netdata}\footnote{\url{https://github.com/firehol/netdata}}%
~\footnote{\url{http://london.my-netdata.io/default.html}}
permite monitorear en tiempo real una serie de datos del sistema. Para ello
cuenta con distintos plugins que implementan el acceso a la información y el
método para graficarla. Es posible desarrollar plugins para netdata utilizando
diferentes lenguajes: BASH, Python, Perl, node.js, Java, Go, Ruby, entre otros.
También, es posible implementar plugins binarios con C. En este documento se
describe como crear un plugin en C utilizando la API para plugins externos.

El primer paso será implementar una System Call de forma similar a la generada
en la práctica 3. Luego será necesario escribir un plugin para netdata que
utilice esta system call y permita graficar sus datos con esta herramienta.
