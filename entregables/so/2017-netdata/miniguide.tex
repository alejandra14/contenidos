\section{Miniguía para crear la system call}

De forma similar a la práctica se recomienda agregar la implementación de
la system call a un archivo
existente del kernel según sea conveniente. Para este caso una posibilidad es
agregarla en \texttt{net/socket.c} ya que este archivo incluye los archivos
\texttt{.h} necesarios para acceder a los dispositivos de red.

Serán de interés para esta práctica los siguientes tipos declarados en\\
\texttt{include/linux/netdevice.h}:

\begin{description}
    \item [struct net\_device]: Cada interfaz de red tiene asociada una
    estructura de este tipo.
    \item [struct net\_device\_stats]: Dentro de cada
        estructura \verb+struct net_device+ un campo \verb+stats+ contiene una
        estructura de este tipo con todos los datos necesarios para este
        trabajo.
\end{description}

Investigar estos tipos de datos mencionados, la estructura \textit{init\_net} y
las macros \texttt{first\_net\_device} y \texttt{next\_net\_device}. Estos
serán necesarios para obtener la información pedida por el enunciado.

Considerar que el kernel cuenta con sus propias implementaciones de
\texttt{strcmp} y \texttt{strncmp} que pueden resultar útiles a la hora de
identificar \textit{eth0}.

Se recomienda que la system call a implementar reciba como argumento un puntero
a una estructura de tipo \verb+“struct net_device_stats”+ donde se cargarán los
datos de la interfaz en caso de éxito.

Nota 1: Al declarar el prototipo de la system call en syscalls.h será necesario
incluir “linux/netdevice.h”.

Nota 2: Usar un sistema de la misma
arquitectura que el kernel (todo de 32 bits o todo de 64 bits). De otra manera
se complejiza la copia de las estructuras de datos desde el kernel hacia el
espacio de usuario y la declaración de la system call.

\section{Miniguía para crear plugins de netdata}

Es posible crear plugins para netdata en cualquier lenguaje, la forma de enviar
la información es escribiendo en stdout. Sin embargo, en esta práctica es
necesario invocar a una System Call personalizada, por lo que la forma más
directa de hacerlo será usando el lenguaje C.

Cada plugin será ejecutado como un programa normal, una sola vez cuando netdata
arranca, se espera que al arrancar el programa defina los gráficos para los que
proveerá datos (CHARTs) y las curvas o datos que proveerá para cada gráfico
(DIMENSIONs). Una vez iniciado, el programa deberá ejecutarse continuamente
imprimiendo datos a intervalos regulares.

El siguiente es un ejemplo de un plugin externo en C que grafica datos
aleatorios:
\lstinputlisting[frame=lrtb,language=C]{ejemplo_plugin.c}


Adicionalmente es posible inspeccionar otros plugins de ejemplo en el código
fuente de netdata.

Para instalar el plugin alcanza con compilarlo con un nombre que termine en
“.plugin” y guardarlo en el directorio \texttt{/usr/libexec/netdata/plugins.d}
(si es que se instaló netdata en el directorio raíz). Luego, será necesario
reiniciar netdata:

\begin{lstlisting}[frame=lrtb]
killall netdata; sleep 0.5; netdata
\end{lstlisting}

Nota: Si se utiliza una estructura de tipo \verb+struct net_device_stats+
para copiar los datos al espacio de usuario será
necesario copiar su declaración al plugin (no es posible incluir
\texttt{linux/netdevice.h} desde un programa de usuario sin mayores problemas).
\section{Más documentación y referencias útiles}


\begin{itemize}
    \item \url{https://github.com/firehol/netdata/tree/master/plugins.d}
    \item \url{https://github.com/firehol/netdata/wiki/External-Plugins}
    \item \url{https://my-netdata.io/} y \url{https://github.com/firehol/netdata}
    \item \url{https://www.ibm.com/developerworks/library/l-kernel-memory-access/}
    \item \url{https://www.kernel.org/doc/html/latest/process/adding-syscalls.html}
\end{itemize}


