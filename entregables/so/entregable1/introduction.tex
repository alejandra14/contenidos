Tal y como hemos estudiado con anterioridad, una shell provee una interfaz que permite al usuariocomunicarse con el \textit{kernel}. La estructura de esta interfaz requiere un interesante trabajo de ingeniería puesto que se deben tener en cuenta aspectos tales como la comunicación entre procesos, redirecciones, tareas en \textit{background} o \textit{foreground}, estructuras de control, entre otras características.

Sin duda alguna, las \textit{shells} más utilizadas en la actualidad son aquellas basadas en la \textit{Bourne shell}, también conocida como \textbf{sh}. Este tipo de \textit{shells} tienen la estructura que hemos estudiado en esta materia, donde las redirecciones se hacen con los símbolos \textbf{>} y \textbf{<}, los pipes se representan con \textbf{|}, los procesos se ejecutan en background con \textbf{\&}, etc. Sin embargo, el alumno debe entender que esta es solo una de las tantas interfaces que se pueden diseñar para interactuar con el sistema operativo.

Con el objetivo de ilustrar este concepto, el presente trabajo propone el desarrollo de una \textit{shell}, limitada, con una interfaz distinta a la estudiada. También se planteará una peculiar pre-condición: la \textit{shell} debe ser desarrollada utilizando \textbf{bash}.



