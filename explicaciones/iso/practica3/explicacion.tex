\section{Introducción}
\begin{frame}
  \frametitle{Introducción}
  ¿Qué es una shell?
  \begin{itemize}
  \item Intérprete de \alert{comandos}
  \item Interactivo
  \item En sistemas operativos *nix es configurable
  \item Proveen estructuras de control que permiten programar \textit{shell scripts}
  \end{itemize}

  \pause
  ¿Qué puedo hacer con \textit{shell scripts}?
  \begin{itemize}
  \item \alert{Automatización} de tareas
  \item Aplicaciones interactivas
  \item Aplicaciones con interfaz gráfica (con el comando \textit{zenity},
    por ejemplo)
  \end{itemize}
\end{frame}

\begin{frame}{Tipos de \textit{shell}}
  Existen muchas \textit{shells}. Sus diferencias consisten principalmente
  en sintaxis. A continuación se listan las más utilizadas:
  \begin{itemize}
  \item \texttt{sh:} \textit{Shell} por defecto en Unix.
  \item \texttt{bash:} Cómoda, instalada por defecto en la mayoría de las
    distribuciones.
  \item \texttt{dash:} Eficiente, parcialmente compatible con
    \texttt{bash}.
  \item \texttt{csh:} Sintaxis incompatible con
    \texttt{bash}/\texttt{dash}.
  \item Otros...
  \end{itemize}
  \pause
  \begin{block}{Tip}
	En la materia utilizaremos \texttt{bash}.
  \end{block}
\end{frame}

\begin{frame}
	\frametitle{Diferencias con otros lenguajes}
	¿Por qué \textit{shell script} y no C, o Java, o Python?
	\begin{itemize}
		\item Práctico para manejar archivos
		\item Extremadamente \alert{simple} para crear procesos y manipular
          sus salidas
		\item \alert{Independiente} de la plataforma (a diferencia de C)
		\item Funciona en cualquier sistema operativo de tipo *nix
          (distribución GNU/Linux, Mac OS X, etc.)
		\item Se puede probar en el \alert{intérprete interactivo} (a
          diferencia de C y Java)
	\end{itemize}
\end{frame}

\begin{frame}
	\frametitle{Elementos del lenguaje}
	\begin{itemize}
		\item Instrucciones: comandos
		\begin{itemize}
			\item Internos o \textit{built-in} (\texttt{help} para verlos)
			\item Externos (archivos separados \texttt{man comando})
		\end{itemize}
		\item Redirecciones y \textit{pipes}
		\item Comentarios que empiezan con \texttt{\#}
		\item Estructuras de control
		\begin{itemize}
			\item \texttt{if}
			\item \texttt{while}
			\item \texttt{for} (2 tipos)
			\item \texttt{case}
		\end{itemize}
		\item Variables
		\begin{itemize}
			\item \textit{Strings}
			\item Arreglos ()
		\end{itemize}
		\item Funciones
	\end{itemize}
\end{frame}

\section{Conceptos básicos}
\subsection{Comandos}
\begin{frame}[fragile]
	\frametitle{Repaso de algunos comandos útiles}
	\begin{itemize}
		\item Imprimir el contenido de un archivo
		\begin{lstlisting}
			cat archivo
		\end{lstlisting}
		\item Imprimir texto
		\begin{lstlisting}
			echo "Hola mundo"
		\end{lstlisting}
		\item Leer una línea desde entrada estándar en la variable \texttt{var}
		\begin{lstlisting}
			read var
		\end{lstlisting}
      \item Quedarme con la primer columna de un texto separado por
        \texttt{:} desde entrada estándar
		\begin{lstlisting}
			cut -d: -f1
		\end{lstlisting}
		\item Contar la cantidad de líneas que se leen desde entrada estándar
		\begin{lstlisting}
			wc -l
		\end{lstlisting}
	\end{itemize}
\end{frame}

\begin{frame}[fragile]
	\frametitle{Repaso de algunos comandos útiles}
	\begin{itemize}
    \item Buscar todos los archivos que contengan la cadena \textbf{pepe}
      en el directorio \texttt{/tmp}
		\begin{lstlisting}
			grep pepe /tmp/*
		\end{lstlisting}
      \item Buscar todos los archivos dentro del \textit{home} del usuario,
        cuyo nombre termine en \texttt{.doc}
		\begin{lstlisting}
			find $HOME -name "*.doc"
		\end{lstlisting}%$
		
		\item Buscar todos los archivos dentro del directorio actual que sean enlaces simbólicos
		\begin{lstlisting}
			find -type l
		\end{lstlisting}
	\end{itemize}
\end{frame}

\begin{frame}[fragile]
	\frametitle{Repaso de algunos comandos útiles}
	\begin{itemize}
		\item \textbf{Empaquetado:} Se unen varíos archivos en uno solo (\texttt{tar})
		\begin{lstlisting}
			tar –cvf archivo.tar archivo1 archivo2 archivo 3
			tar –xvf archivo.tar
		\end{lstlisting}
		\item \textbf{Compresión:} Se reduce el tamaño de un archivo (\texttt{gzip}/\texttt{bzip2}/etc.)
		\begin{lstlisting}
			gzip archivo.tar # Genera archivo.tar.gz comprimido
			gzip –d archivo.tar.gz # Descomprime archivo.tar
		\end{lstlisting}
		\item El comando \texttt{tar} puede invocar a \texttt{gzip} por nosotros (argumento ``z''):
		\begin{lstlisting}
			tar –cvzf archivo.tar.gz arch1 arch2 arch3
			tar –xvzf archivo.tar.gz
		\end{lstlisting}	
	\end{itemize}
\end{frame}

\subsection{Redirecciones y pipes}
\begin{frame}
	\frametitle{Redirecciones y pipes: stdin, stdout, stderr}
	Los procesos (programas en ejecución) normalmente cuentan con 3 ``archivos'' abiertos.
	\begin{itemize}
		\item \textbf{stdin:} Entrada estándar, normalmente el teclado.
		\item \textbf{stdout:} Salida estándar, normalmente el monitor.
		\item \textbf{stderr:} Error estándar, normalmente la salida
          estándar.
	\end{itemize}
	Se identifican en el S.O. con un número, el \alert{\textit{file
        descriptor}} (descriptor de archivo):
	\begin{enumerate}
		\setcounter{enumi}{-1}
		\item Entrada estándar
		\item Salida estándar
		\item Error estándar
	\end{enumerate}
\end{frame}

\begin{frame}[fragile]
	\frametitle{Redirecciones}
	Sintaxis básica:
	\begin{lstlisting}
		comando > archivo
		comando >> archivo
	\end{lstlisting}
	\begin{itemize}
		\item \textbf{\textgreater} Redirección destructiva:
		\begin{itemize}
			\item Si \textit{archivo} no existe, se crea.
			\item Si \textit{archivo} existe, sobreescribe.
		\end{itemize}
		\item \textbf{\textgreater\textgreater} Redirección no destructiva:
		\begin{itemize}
			\item Si \textit{archivo} no existe, se crea.
			\item Si \textit{archivo} existe, agrega al final.
		\end{itemize}
	\end{itemize}
    \pause
	¿Qué hace?:
	\begin{lstlisting}
		cd
		ls >> /tmp/lista.txt
		cd /tmp
		ls >> /tmp/lista.txt
	\end{lstlisting}

	¿Y si cambiamos \textgreater\textgreater{} por \textgreater{}?
\end{frame}

\begin{frame}[fragile]
	\frametitle{Otras redirecciones}
	Sintaxis básica:
	\begin{lstlisting}
		comando 2> archivo
		comando 2>> archivo
		comando < archivo
	\end{lstlisting}
	\begin{itemize}
		\item \textbf{2\textgreater{} y 2\textgreater\textgreater} Redirigen el error estándar
		\item \textbf{\textless} Hace que \textit{archivo} sea la entrada de \textit{comando}.
		
		En otras palabras cuando \textit{comando} intente leer entrada del teclado, en realidad, va a leer el contenido de \textit{archivo}.
	\end{itemize}
\end{frame}

\begin{frame}[fragile]
	\frametitle{Pipes (tuberías)}
	Sintaxis básica:
	\begin{lstlisting}
		comando | comando2 | comando3
	\end{lstlisting}
	\begin{itemize}
    \item Conectan la \alert{salida} de un comando con la \alert{entrada}
      de otro.
		\item Indispensables para hacer programas potentes en \textit{shell
            script}.
	\end{itemize}
	Ejemplos:
	\begin{lstlisting}
		cat archivo | tr a-z A-Z
		cat archivo | grep hola | cut -d, -f1
		cat /etc/passwd | cut –d: -f1 | grep a | wc -l
		cat /etc/passwd | cut -d: -f7 | sort | uniq > res.txt
	\end{lstlisting}	
\end{frame}


\subsection{Variables y sustitución de comandos}
\begin{frame}[fragile]
	\frametitle{Variables}
	\begin{itemize}[<+->]
		\item \texttt{bash} soporta \textit{strings} y \textit{arrays}
		\item Los nombres son \textit{case sensitive}
		\item Para crear una variable:
		\begin{lstlisting}
			NOMBRE="pepe" # SIN espacios alrededor del =
		\end{lstlisting}
		\item Para accederla se usa \${}:
		\begin{lstlisting}
			echo $NOMBRE
		\end{lstlisting}%$
		
		\item Para evitar ambiguedades se pueden usar llaves:
		\begin{lstlisting}
			# Esto no accede a $NOMBRE
			echo $NOMBREesto_no_es_parte_de_la_variable
			# Esto sí
			echo ${NOMBRE}esto_no_es_parte_de_la_variable
		\end{lstlisting}%$
	\end{itemize}
\end{frame}

\begin{frame}[fragile]
	\frametitle{Variables: Ejemplo 1}
Los nombres de las variables pueden contener mayúsculas, minúsculas, números y el símbolo \_ (underscore)‏, pero no pueden empezar con un número.
	\begin{lstlisting}
		NOMBRE="Fulano De Tal"
		facultad=Informatica
		carrera_1="Licenciatura en Sistemas"
		carrera_2="Licenciatura en Informatica"
		echo El alumno $NOMBRE de la Facultad de $facultad cursa $carrera_1 y $carrera_2
		# imprime:
		#  El alumno Fulano De Tal de la Facultad de 
		#  Informática cursa Licenciatura en Sistemas
		#  y Licenciatura en Informatica
	\end{lstlisting}
\end{frame}
\begin{frame}[fragile]
	\frametitle{Variables: Ejemplo 2}
	\begin{lstlisting}
		nombre=Carlos
		echo "Hola $nombre" # Hola Carlos
		echo Hola ${nombre} # Hola Carlos
		nombre=5
		echo "Hola $nombre" # Hola 5
	\end{lstlisting}%$
\end{frame}

\begin{frame}[fragile]
  \frametitle{Variables: Arreglos}
  \begin{itemize}
  \item \alert{\textit{``Bashismo''}}
  \item Creación:
    \begin{lstlisting}
      arreglo_a=() # Se crea vacío
      arreglo_b=(1 2 3 5 8 13 21) # Inicializado
    \end{lstlisting}
  \item Asignación de un valor en una posición concreta:
    \begin{lstlisting}
      arreglo_b[2]=spam
    \end{lstlisting}
  \item Acceso a un valor del arreglo (En este caso las llaves no son opcionales):
    \begin{lstlisting}
      echo ${arreglo_b[2]}
      copia=${arreglo_b[2]}
    \end{lstlisting}
  \item Acceso a todos los valores del arreglo:
    \begin{lstlisting}
      echo ${arreglo[@]} # o bien ${arreglo[*]}
    \end{lstlisting}
  \end{itemize}
\end{frame}
\begin{frame}[fragile]
	\frametitle{Variables: Arreglos}
	\begin{itemize}
		\item Tamaño del arreglo:
			\begin{lstlisting}
				${#arreglo[@]} # o bien ${#arreglo[*]}
			\end{lstlisting}
		\item Borrado de un elemento (reduce el tamaño del arreglo pero no elimina la posición, solamente la deja vacía):
			\begin{lstlisting}
				unset arreglo[2]
			\end{lstlisting}
		\item Los índices en los arreglos comienzan en 0
	\end{itemize}
\end{frame}
\begin{frame}[fragile]
	\frametitle{Variables: Ejemplo de arreglos}
	\begin{lstlisting}
#!/bin/bash

arreglo=(1 2 3 5 8 13 21)‏
arreglo[2]=spam
echo "El primer elemento es ${arreglo[0]}"
echo "El tercer elemento es ${arreglo[2]}"
echo "La longitud: ${#arreglo[*]}"
echo "Todos sus elementos: ${arreglo[*]}"
\end{lstlisting}
\end{frame}

\begin{frame}[fragile]
  \frametitle{Variables: Comillas}
  \begin{itemize}
  \item No hacen falta, a menos que:
    \begin{itemize}
    \item el \textit{string} tenga espacios.
    \item que sea una variable cuyo contenido pueda tener espacios.
    \item son importantes en las condiciones de los \texttt{if}, \texttt{while}, etc...
    \end{itemize}
  \item Tipos de comillas
    \begin{itemize}
    \item \textbf{``Comillas dobles'':}
      \begin{lstlisting}
var='variables'
echo "Permiten usar $var"
echo "Y resultados de comandos $(ls)"
     \end{lstlisting}
    \item \textbf{`Comillas simples':}
      \begin{lstlisting}
echo 'No permiten usar $var'
echo 'Tampoco resultados de comandos $(ls)'
      \end{lstlisting}
    \end{itemize}
  \end{itemize}
\end{frame}

\begin{frame}[fragile]
	\frametitle{Comillas: Ejemplos}
	\begin{lstlisting}
		Un ejemplo:

		variable="un texto de varias palabras"
		variable_2=UnaSolaPalabra

		echo "Podemos leer $variable"
		echo 'No podemos leer $variable'

		variable_3="Asi concateno $variable_2 a otro string"
	\end{lstlisting}
	\begin{itemize}
		\item ¿Qué se imprime en cada caso?
		\item ¿Cuál es el valor de variable\_3?
	\end{itemize}
\end{frame}
\subsection{Reemplazo de comandos}
\begin{frame}[fragile]
	\frametitle{Reemplazo de comandos}
	\begin{itemize}
		\item Permite utilizar la salida de un comando como si fuese una cadena de texto normal.
		\item Permite guardarlo en variables o utilizarlos directamente.
		\item Se la puede utilizar de dos formas, cada una con distintas reglas:
		\begin{lstlisting}
			$(comando_valido)‏
			`comando_valido`
		\end{lstlisting}
		\textbf{Nota:} La primer forma resulta más clara y posee reglas de anidamiento de comandos más sencillas.
		\item Ejemplo:
		\begin{lstlisting}
			arch="$(ls)" # == arch="`ls`" == arch=`ls`
			mis_archivos="$(ls /home/$(whoami))"
		\end{lstlisting}
	\end{itemize}
\end{frame}

\section{Programación de scripts}
\subsection{Scripts}
\begin{frame}[fragile]
  \frametitle{¿Cómo hacer un script?}
  \begin{itemize}
  \item Crear un archivo con cualquier editor de texto.
    \pause
  \item Indicar el intérprete (opcional) en la primer línea con:
    \begin{lstlisting}
      #!/bin/bash
    \end{lstlisting}
    Está línea, denominada \textit{shebang}, especifica el intérprete
    que se utilizará para ejecutar \textit{script}. Si no se
    especifica, se utiliza el intérprete por defecto del usuario.
    \begin{itemize}
      \pause
    \item   ¿Qué problemas puede traer esto?
    \end{itemize}
    \pause
  \item Escribir una serie de comandos en el archivo, de la misma manera
    que los escribiriamos en la terminal.
    \pause
  \item ¿Guardar el script con terminación \textit{.sh}?
    \pause
  \item ¿Permisos de ejecución?
  \end{itemize}
\end{frame}

\begin{frame}[fragile]
	\frametitle{Un primer script}
	\begin{block}{Ejemplos}
		\begin{lstlisting}
			#!/bin/bash
			 
			# Si la primer línea de mi script comienza
			# con  la cadena #! se interpretará como el
			# path al intérprete a utilizar (podría ser
			# python, perl, php, etc...)‏

			# Ahora el script en sí
			echo "Hola mundo"
		\end{lstlisting}
	\end{block}
\end{frame}

\begin{frame}
	\frametitle{Ejecución del script}
	Repaso de conceptos:
	\begin{itemize}
		\item \textbf{Path absoluto:} Empieza desde el directorio raíz /
		\item \textbf{Path relativo:} Empieza desde donde estamos posicionados
		\item \textbf{. y ..} Directorio actual y directorio padre respectivamente
		\item \textbf{Variable de entorno PATH}.
	\end{itemize}

    \pause

    ¿Cómo ejecutar un \textit{script}?
	\begin{itemize}[<+->]
		\item Lo guardamos
		\item ¿Le damos permisos de ejecución? 
		\item Lo ejecutamos 
		\begin{itemize}
			\item \texttt{./mi\_script.sh}
			\item \texttt{bash mi\_script.sh}
			\item O podemos también ejecutarlo en modo \textit{debug} (depuración)\\
			\texttt{bash -x mi\_script.sh}
		\end{itemize}
	\end{itemize}
\end{frame}

\subsection{Estructuras de control}
\begin{frame}[fragile]
	\frametitle{Estructuras de control}
	\begin{block}{Selección de alternativas:}
	\begin{columns}
	    \begin{column}{.45\linewidth}
		Decisión:
	    	\begin{lstlisting}
			if [ condition ]
			then
			  block
			fi
		\end{lstlisting}
	    \end{column}
	    \begin{column}{.45\linewidth}
		Selección:
	    	\begin{lstlisting}
			case $variable in
			  "valor 1")
			    block
			    ;;
			  "valor 2")
			    block
			    ;;
			  *)
			    block
			    ;;
			esac
		\end{lstlisting}
	    \end{column}
	\end{columns}
	\end{block}
\end{frame}

\begin{frame}[fragile]
	\frametitle{Estructuras de control}
	\begin{block}{Menú de opciones:}
		\begin{lstlisting}
			select variable in opcion1 opcion2 opcion3
			do
			  # en $variable está el valor elegido
			  block
			done
		\end{lstlisting}
	\end{block}
\end{frame}
\begin{frame}[fragile]
	\frametitle{Estructuras de control}
	\begin{block}{Menú de opciones:}
	\begin{columns}
		\begin{column}{.50\linewidth}
			Ejemplo:
			\begin{lstlisting}
				select action in New Exit
				do
				  case $action in
				    "New")‏
				      echo "Selected option is NEW" 
				      ;;
				    "Exit")‏
				      exit 0
				      ;;
				  esac
				done
			\end{lstlisting}
		\end{column}
		\begin{column}{.45\linewidth}
			Imprime:
			\begin{Verbatim}
				1) new
				2) exit
				#?
			\end{Verbatim}
			y espera el número de opción por teclado
		\end{column}
	\end{columns}
	\end{block}
\end{frame}
\begin{frame}[fragile]
	\frametitle{Iteración - bloques \texttt{for}}
	\begin{itemize}
		\item C-\textit{style}:
		\begin{lstlisting}
			for ((i=0; i < 10; i++))
			do
			  block
			done
		\end{lstlisting}
		\item Con lista de valores (\texttt{foreach}):
		\begin{lstlisting}
			for i in value1 value2 value3 valueN;
			do
			  block
			done
		\end{lstlisting}
	\end{itemize}
\end{frame}
\begin{frame}[fragile]
	\frametitle{Iteración - Bloques WHILE y UNTIL:}
	\begin{block}{while}
		\begin{lstlisting}
			while [ condition ] #Mientras se cumpla la condición
			do
			  block
			done
		\end{lstlisting}
	\end{block}
	\begin{block}{until}
		\begin{lstlisting}
			until [ condition ] #Mientras NO se cumpla la condición 
			do
			  block
			done
		\end{lstlisting}
	\end{block}
\end{frame}
\subsection{Comparaciones}
\begin{frame}[fragile]
	\frametitle{Evaluación de condiciones lógicas:}
	Las condiciones lógicas normalmente se evalúan mediante:
	\begin{lstlisting}
		[ condition ]
		test condition
	\end{lstlisting}
	Operadores para condition:
	\begin{tabular}{|l|c|c|}
		\hline
		\textbf{Operador} & \textbf{Con strings} & \textbf{Con números} \\ \hline
		Igualdad & \verb+"$nombre" = "Maria"+ & \verb+$edad -eq 20+ \\ \hline
		Desigualdad & \verb+"$nombre" != "Maria"+ & \verb+$edad -ne 20+ \\ \hline
		Mayor & \verb+A > Z+ & \verb+5 -gt 20+ \\ \hline
		Mayor o igual & \verb+A >= Z+ & \verb+5 -ge 20+ \\ \hline
		Menor & \verb+A < Z+ & \verb+5 -lt 20+ \\ \hline
		Menor o igual & \verb+A <= Z+ & \verb+5 -le 20+ \\ \hline
	\end{tabular}

\end{frame}
\subsection{Estructuras de control en detalle}
\begin{frame}[fragile]
	\frametitle{Estructuras de control - ejemplos:}
	\begin{block}{Ejemplos:}
	\begin{columns}
		\begin{column}{.45\linewidth}
			\begin{lstlisting}[basicstyle=\tiny]
				if [ "$USER" == root ]
				then
				    echo "super user"
				else
				    echo "Ud. es $USER"
				fi
				# -------------------
				n=0
				while [ $n -ne 5 ]; do
				  echo $n 
				  let n++
				done
			\end{lstlisting}
		\end{column}
		\begin{column}{.45\linewidth}
			\begin{lstlisting}[basicstyle=\tiny]
				for archivo in $(ls)‏
				do
				  echo "- $archivo"
				done
				# -------------------
				for ((i=0; i<5; i++))‏
				do
				  echo $i
				done
			\end{lstlisting}
		\end{column}
	\end{columns}
	\end{block}
    \pause
	Adicionalmente:
	\begin{itemize}
    \item \texttt{break [n]} corta la ejecución de n niveles de
      \textit{loops}.
    \item \texttt{continue [n]} salta a la siguiente iteración del enésimo
      \textit{loop} que contiene esta instrucción.
	\end{itemize}
\end{frame}

\begin{frame}[fragile]
	\frametitle{Estructuras de control - break}
	\begin{lstlisting}
		#!/bin/bash
		# Imprime los numeros del 1 al 5
		# (no es un código para nada elegante)‏
		i=0
		# true es un comando que siempre retorna 0
		while true
		do
		  let i++   # Incrementa i en 1
		  if [ $i -eq 6 ]; then
		      break   # Corta el loop (while)‏
		  fi
		  echo $i
		done
	\end{lstlisting}
\end{frame}
\begin{frame}[fragile]
	\frametitle{Estructuras de control - break}
	\begin{center}
		{\Large ¿Qué hace el siguiente \textit{script}?}
	\end{center}

	\begin{lstlisting}
		#!/bin/bash
		i=0 
		while true
		do
		    let i++
		    if [ $i -eq 6 ]; then
		        break    # Corta el while
		    elif [ $i -eq 3 ]; then
		        continue # Salta una iteración
		    fi
		    echo $i  
		done
	\end{lstlisting}
\end{frame}
\begin{frame}[fragile]
	\frametitle{Condiciones compuestas}
	\begin{lstlisting}
		# AND
		if [ $a = $b ] && [ $a = $c ]
		then
		  #...
		 
		# OR
		if [ $a = $b ] || [ $a = $c ]
		then
		  #...
	\end{lstlisting}
\end{frame}
\subsection{Argumentos y valor de retorno}
\begin{frame}[fragile]
	\frametitle{Argumentos y valor de retorno}
	\begin{itemize}
		\item Los \textit{scripts} pueden recibir argumentos en su invocación.
		\item Para accederlos, se utilizan variables especiales:
		\begin{itemize}
			\item \texttt{\$0} contiene la invocación al script.
			\item \texttt{\$1}, \texttt{\$2}, \texttt{\$3}, ... contienen cada uno de los argumentos.
			\item \texttt{\$\#} contiene la cantidad de argumentos recibidos.
			\item \texttt{\$*} contiene la lista de todos los argumentos.
			\item \texttt{\$?} contiene en todo momento el valor de retorno del último comando ejecutado.
		\end{itemize}
	\end{itemize}
	Ejemplo:
	\begin{lstlisting}
		if [ $# -ne 2 ]; then
		    exit 1 # Error
		else
		    echo "Nombre: $1, Apellido: $2"
		fi
		exit 0     # Funcionó correctamente
	\end{lstlisting}
\end{frame}

\begin{frame}[fragile]
	\frametitle{Terminación de un \textit{script}}
    Para terminar un \textit{script} usualmente se utiliza la función \texttt{exit}:
	\begin{itemize}
		\item Causa la terminación de un \textit{script}
		\item Puede devolver cualquier valor entre 0 y 255:
		\begin{itemize}
			\item El valor 0 indica que el script se ejecutó de forma exitosa
			\item Un valor distinto indica un código de error
			\item Se puede consultar el \textit{exit status} imprimiendo la
              variable \texttt{\$?}
		\end{itemize}
	\end{itemize}
\end{frame}

\subsection{Funciones}
\begin{frame}[fragile]
	\frametitle{Funciones}
    Las funciones permiten modularizar el comportamiento de los \textit{scripts}.
	\begin{itemize}
		\item Se pueden declarar de 2 formas:
          \begin{itemize}
		\item \verb+function nombre { block }+
		\item \verb+nombre() { block }+
        \end{itemize}
		\item Con la sentencia \texttt{return} se retorna un valor entre 0 y 255
		\item El valor de retorno se puede evaluar mediante la variable \texttt{\$?}
		\item Reciben argumentos en las variables \texttt{\$1},
          \texttt{\$2}, etc.
	\end{itemize}
\end{frame}

\begin{frame}[fragile]
	\frametitle{Funciones: ejemplos}
	\begin{lstlisting}
		# Recibe 2 argumentos y devuelve:
		#   1 si el primero es el mayor
		#   0 en caso contrario
		mayor() {
		  echo "Se van a comparar los valores $*" 
		  if [ $1 -gt $2 ]; then
		    echo "$1 es el mayor" 
		    return 1
		  fi
		  echo "$2 es el mayor"
		  return 0
		}
		mayor 5 6 # Invocación
		echo $?   # Imprime el exit Status de la funcion
	\end{lstlisting}
\end{frame}
\subsection{Alcance y visibilidad}
\begin{frame}[fragile]
	\frametitle{Variables: alcance y visibilidad}
	\begin{itemize}
		\item Las variables no inicializadas son reemplazadas por un valor nulo o 0, según el contexto de evaluación.
		\item Por defecto las variables son globales.
		\item Una variable local a una función se define con \texttt{local}
		\begin{lstlisting}
			test() {
			  local variable
			}
		\end{lstlisting}
		\item Las variables de entorno son heredadas por los procesos hijos.
		\item Para exponer una variable global a los procesos hijos se usa
          el comando \texttt{export}:
		\begin{lstlisting}
			export VARIABLE_GLOBAL="Mi var global"
			comando
			# comando verá entre sus variables de
			# entorno a VARIABLE_GLOBAL
		\end{lstlisting}
	\end{itemize}
\end{frame}

