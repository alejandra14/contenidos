\section{Propuesta}

\begin{frame}{Propuesta de contenidos}
  \begin{itemize}
  \item La idea general de esta propuesta es definir nuevo contenido para
    la cátedra.
    
  \item Dejar de abarcar tantos sistemas operativos puesto que el
    conocimiento que los alumnos adquieren es superficial.
    
  \item Motivar a los alumnos con contenido técnico y desafiante. Que las
    prácticas no sean tan guiadas.
    
  \item Profundizar en un sistema operativo, en particular *nix.
    
  \item Introducir una serie de ejercicios entregables, tal vez en grupo,
    con el objetivo de generar un seguimiento del avance de los alumnos.
    
\end{itemize}
\end{frame}

\subsection{Práctica 1}
\begin{frame}{Práctica 1}
\begin{itemize}
\item Repaso comandos
\item \textit{Pipes}
\item Redirecciones
\item \texttt{netcat} (para jugar un poco con redes y redirecciones)
\item \textit{Scripts} de repaso
\item Ejercicio entregable, shell 1.
\end{itemize}
\end{frame}

\subsection{Práctica 2}
\begin{frame}{Práctica 2}
\begin{itemize}
\item Compilación \textit{kernel}
\item Ejercicio entregable, shell 2.
\end{itemize}
\end{frame}

\subsection{Práctica 3}
\begin{frame}{Práctica 3}
\begin{itemize}
\item \textit{System calls}, módulos y \textit{drivers}
\item Ejercicio entregable, shell 3.
\end{itemize}
\end{frame}

\subsection{Práctica 4}
\begin{frame}{Práctica 4}
  Android:
  \begin{itemize}
  \item Dejar de usar \texttt{Geanymotion}, ni ningún otro IDE,
    para dar lugar a las herramientas que provee el SDK.
  \item rooteo
  \item etc.
  \end{itemize}
\end{frame}  
\subsection{Práctica 5}
\begin{frame}{Práctica 5}
  CPU y memoria
  \begin{itemize}
  \item \textit{How programs get run}
  \item CPU \textit{scheduling}: multi-processor
  \item Tablas de paginación.
  \end{itemize}
\end{frame}  

\subsection{Práctica 6}
\begin{frame}{Práctica 6}
  \textit{File systems}
  \begin{itemize}
  \item \texttt{/proc}
  \item \texttt{sysfs} y \texttt{/sys} (\texttt{kobject solo si se
      necesita ;-(}.
  \item Visión completa FS Linux en base a \texttt{ext4} (i-nodos, VFS,
    etc.).
    
  \item Algún ejercicio programable utilizando FUSE (Filesystem in Userspace).
  \end{itemize}
\end{frame}  

\section{Forma de trabajo}
\begin{frame}{\texttt{git} y \LaTeX}
  \begin{itemize}
  \item Repositorio común para las materias.
  \item Versionado \texttt{git}.
  \item Cualquier integrante de la cátedra puede colaborar. Incluso alumnos
    podrían mandar \textit{pull requests}.
  \item Beneficios:
    \begin{itemize}
    \item Historial de todas las prácticas
    \item Registro de modificaciones
    \item Control de estilo de prácticas y explicaciones en un único punto.
    \item Generación de prácticas y explicaciones de las dos materias con
      un solo comando (\texttt{make}).
    \item Todo el poder de \texttt{git} y \LaTeX. ;-)
    \end{itemize}
  \end{itemize}
\end{frame}

%%% Local Variables: 
%%% mode: latex
%%% TeX-master: "main"
%%% End: