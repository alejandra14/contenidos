\section{Pipes y Netcat}
\subsection{Pipes}

\subsubsection{Unnamed Pipes}
\begin{frame}[fragile]
  \frametitle{Unnamed Pipes}
  \begin{itemize}
    \item Son los que se utilizaron para realizar los scripts en ISO (\texttt{|})
    \item ¿Cuántos procesos actuan en la siguiente linea ejecutada en una terminal? (\textit{subshell\footnote{\url{http://www.tldp.org/LDP/abs/html/subshells.html}}})
    \begin{lstlisting}
ls | grep x
    \end{lstlisting}
    \pause
    \begin{lstlisting}
echo "Subshell_1 \$BASH_SUBSHELL=$BASH_SUBSHELL \$BASHPID=$BASHPID \$\$=$$ \$PPID=$PPID" | (read v; echo "$v - Subshell_2 \$BASH_SUBSHELL=$BASH_SUBSHELL \$BASHPID=$BASHPID \$\$=$$ \$PPID=$PPID")
    \end{lstlisting}
  \end{itemize}
\end{frame}

\subsubsection{Named Pipes}
\begin{frame}[fragile]
  \frametitle{Named Pipes}
  \begin{itemize}
    \item También conocidas con el nombre de \textit{FIFO}
    \item Se representan mediante archivos
    \item Se crean con el comando \textit{mkfifo} o \textit{mknode}
    \item Se identifican con la letra \textit{p}
    \begin{lstlisting}
$ ls -l
prw-r--r--   1 root root    0 Jan 22 23:11 named_pipe
    \end{lstlisting}
  \end{itemize}
\end{frame}

\begin{frame}[fragile]
  \frametitle{Ejemplo con Named Pipes}
  \begin{itemize}
   \item Si en una terminal ejecutamos lo siguiente
   \begin{lstlisting}
mkfifo pipe_so; ls -l > pipe_so
   \end{lstlisting}
   \item Y en otra terminal ejecutamos los siguiente
   \begin{lstlisting}
cat < pipe_so
   \end{lstlisting}
   \item ¿Qué sucede?
  \end{itemize}
\end{frame}

\subsection{Netcat}
\begin{frame}[fragile]
  \frametitle{Netcat}
  \begin{itemize}
    \item Es una utilidad de red destacada
    \item Permite leer y escribir datos a través de conexiones \textit{TCP}
      \begin{lstlisting}
# En el servidor
nc -l port_number

# En el cliente
nc host_name port_number
      \end{lstlisting}
    \item También permite utilizar \textit{UDP} (parametro \textit{-u})
    \item Tunneling
    \item Escaneo de puertos (parametro \textit{-z})
      \begin{lstlisting}	
nc -zv host_name port_number
      \end{lstlisting}
  \end{itemize}
\end{frame}

\subsection{Netcat con Pipes y Redirecciones}
\begin{frame}[fragile]
  \frametitle{Netcat con Pipes y Redirecciones}
  \begin{itemize}
    \item Transferencia de datos, como el contenido de un archivo
      \begin{lstlisting}
# En el servidor      
nc -l -p port_number

# En el cliente
nc port_number -q 1 < /home/so/archivo_a_enviar
      \end{lstlisting}      
    \item ¿Cuál es el comportamiento el siguiente código?
      \begin{lstlisting}
# En el servidor
cat named_pipe | nc -l 1500 > named_pipe

# En el cliente
nc localhost 1500
      \end{lstlisting}
  \end{itemize}
\end{frame}