\section{Repaso}
\subsection{Definiciones}
\begin{frame}
  \frametitle{Definiciones}
  \begin{itemize}
    \item Una \textit{shell} es un intérprete de comandos
    \item Interactiva o \textit{batch}
    \item En *nix: shells configurables e intercambiables
    \begin{itemize}
      \item \texttt{sh}
      \item \texttt{bash} (nos basaremos en esta para la materia)
      \item \texttt{dash}
      \item \texttt{csh}
      \item \texttt{zsh}
      \item ...entre otras...
    \end{itemize}
  \end{itemize}
\end{frame}

\subsection{Elementos de una shell}

\begin{frame}
  \frametitle{Comandos internos}
  \begin{itemize}
    \item Embebidos en la shell
    \item \texttt{man bash} para listarlos
    \item \texttt{help COMANDO} para ver su documentación
    \item Hemos visto en ISO:
    \begin{itemize}
      \item \texttt{cd}, \texttt{source}, \texttt{for}, \texttt{if}, \texttt{while},
        \texttt{continue}, \texttt{break}, \texttt{echo}, \texttt{exit}, \texttt{return}, \texttt{export},
        \texttt{bg}, \texttt{fg}, \texttt{help}, \texttt{kill}, \texttt{let}, \texttt{local},
        \texttt{pwd}, \texttt{read}, \texttt{test} (o \texttt{[ ]}, o \texttt{[[ ]]})
    \end{itemize}
  \end{itemize}
\end{frame}

\begin{frame}
  \frametitle{Comandos externos}
  \begin{itemize}
    \item Cualquier binario que instalemos
    \item ¡Cualquier script que hagamos nosotros mismos!
    \item \texttt{man COMANDO} para ver su documentación -$\rangle$ \texttt{/usr/share/doc} (texto plano, HTML, gzip)
    \item Parámetro \texttt{--help}
    \item Hemos visto en ISO:
    \begin{itemize}
      \item \texttt{grep}, \texttt{cat}, \texttt{cut}, \texttt{ln}, \texttt{ls},
        \texttt{find}, \texttt{expr}, \texttt{rm}, \texttt{mv}, \texttt{cp}, \texttt{whereis}
    \end{itemize}
  \end{itemize}
\end{frame}

\begin{frame}[fragile]
  \frametitle{Comentarios}
  \begin{itemize}
    \item Comienzan con numeral
    \begin{lstlisting}
# Esto es un comentario
ls /etc # Y esto también
    \end{lstlisting}
    \item Son útiles para documentar nuestro código para futura referencia
  \end{itemize}
\end{frame}

\begin{frame}[fragile]
  \frametitle{Estructuras de control}
  \begin{itemize}
    \item Son comandos internos de bash    
    \begin{itemize}
      \item \texttt{if}
      \item \texttt{for}
      \item \texttt{while}
      \item ...
    \end{itemize}
    \item \texttt{if} y \texttt{while} responden a \textit{true} y \textit{false}
    \item Ejemplo:
   \begin{lstlisting}
 if [ 4 -gt 3 ]
 then
   echo 4 es mayor que 3
 else
   echo algo esta roto
 fi
   \end{lstlisting}
  \end{itemize}
\end{frame}

\begin{frame}
  \frametitle{Variables - Strings}
  \begin{itemize}
    \item \textbf{Strings}
    \begin{itemize}
      \item \texttt{hola}
      \item \texttt{\string"hola\string"}
      \item \texttt{\string'hola\string'}
      \item \texttt{\string"string con espacios que reemplaza valores\string"}
      \item \texttt{\string'string con espacios que no reemplaza valores\string'}
    \end{itemize}
  \end{itemize}
\end{frame}

\begin{frame}
  \frametitle{Variables - Arreglos}
  \begin{itemize}
    \item \textbf{Arreglos}
    \begin{itemize}
      \item \texttt{()}
      \item \texttt{(esto es un arreglo con 7 elementos)}
      \item \texttt{(este tiene \string"tres elementos\string")}
    \end{itemize}
  \end{itemize}
\end{frame}

\begin{frame}
  \frametitle{Variables - Asignación y acceso - Strings}
  \begin{itemize}
    \item \textbf{Asignación} (\texttt{=})
    \begin{itemize}
      \item \texttt{mi\_variable=valor}
      \item \texttt{variable=\string'un texto\string'}
      \item \texttt{arreglo=(uno dos tres)}
    \end{itemize}
    \item \textbf{Acceso} (\texttt{\${}})
    \begin{itemize}
      \item \texttt{echo \$mi\_variable}
      \item \texttt{echo \string"imprimo \$variable en pantalla\string"}
      \item \texttt{echo \string'imprimo \$variable en pantalla\string'}
    \end{itemize}
  \end{itemize}
\end{frame}

\begin{frame}
  \frametitle{Variables - Asignación y acceso - Arreglos}
  \begin{itemize}
    \item \textbf{Asignación} (\texttt{=})
    \begin{itemize}
      \item \texttt{arreglo=(1 2 3)}
      \item \texttt{arreglo[3]=cuatro}
      \item \texttt{arreglo[1]=dos}
      \item \texttt{unset arreglo[0]}
    \end{itemize}
    \item \textbf{Acceso} (\texttt{\${}})
    \begin{itemize}
      \item \texttt{echo \$\{arreglo[0]\}}
      \item \texttt{echo \$\{arreglo[*]\}}
      \item \texttt{echo \$\{\#arreglo[*]\}}
    \end{itemize}
  \end{itemize}
\end{frame}

\begin{frame}
  \frametitle{Variables - Alcance}
  \begin{itemize}
    \item Por defecto, todas las variables son \textbf{globales} al script
    \item Se pueden definir variables \textit{locales} mediante \texttt{local}
    \item Existen variables globales a la sesión, llamadas \textbf{variables de entorno}:
    \begin{itemize}
      \item \texttt{export MI\_VARIABLE=\string"su valor\string"}
      \item Ejemplos: \texttt{\$HOME}, \texttt{\$PATH}, \texttt{\$PWD}, \texttt{\$UID}, \texttt{\$EUID}, \texttt{\$IFS}
      \item Se pueden consultar las variables de entorno definidas mediante \texttt{export} o \texttt{env}
      \item Se pueden modificar
      \item Bash utiliza \texttt{\$HOME/.bashrc} para inicializar la shell. Por ejemplo: variable \texttt{\$PATH}, \texttt{\$PS1}, etc.
    \end{itemize}
  \end{itemize}
\end{frame}

\begin{frame}
  \frametitle{Variables Especiales - \textbf{Bash}}
  \begin{itemize}
    \item Son read-only y pueden ser de mucha utilidad:
    \begin{itemize}
      \item \texttt{\$!}: Mantiene el \textit{PID} del último comando ejecutando en \textit{background}.
      \item \texttt{\$\_}: Mantiene el último parámetro del último comando ejecutado.
      \item \texttt{\$\$}: Mantiene el \textit{PID} del proceso actual, es decir el script en si mismo.
    \end{itemize}
  \end{itemize}
\end{frame}

\begin{frame}[fragile]
  \frametitle{Funciones}
  \begin{itemize}
    \item Encapsulan lógica
    \item Una vez definidas, funcionan como nuevos comandos
    \item Para definirlas:
   \begin{lstlisting}
 function imprimir() {
   echo $*
 }

 imprimir Bienvenidos a SO
   \end{lstlisting}
  \end{itemize}
\end{frame}

\begin{frame}[fragile]
  \frametitle{Redirecciones}
  \begin{itemize}
    \item Son unidireccionales
    \item Toman la salida de un comando y la pasan a un archivo
    \item Pueden ser destructivas o no
    \item Por ejemplo:
   \begin{lstlisting}
echo hola > /tmp/salida.txt
echo chau >> /tmp/salida.txt
   \end{lstlisting}
  \end{itemize}
\end{frame}

\begin{frame}[fragile]
  \frametitle{Redirecciones \textit{inversas}}
  \begin{itemize}
    \item Son unidireccionales
    \item Toman el contenido de un archivo y lo pasan a la entrada de un comando
    \item Por ejemplo:
   \begin{lstlisting}
 mysql basededatos < dump.sql
   \end{lstlisting}
  \end{itemize}
\end{frame}

\begin{frame}[fragile]
  \frametitle{Pipes (\texttt{|})}
  \begin{itemize}
    \item Son unidireccionales
    \item Toman la salida de un comando y se la hacen entrada del siguiente
    \item Son un mecanismo de \textbf{IPC}
    \item Utilizan \textit{buffering}
    \item Por ejemplo:
   \begin{lstlisting}
 cat /etc/passwd | cut -d: -f2 | grep a | wc -l
   \end{lstlisting}
  \end{itemize}
\end{frame}

\begin{frame}
  \frametitle{\textit{Shebang}}
  \begin{itemize}
    \item Primer línea (opcional) de un script
    \item Indica qué comando se usará para ejecutar el script
    \item Pueden ser destructivas o no
    \begin{itemize}
      \item \texttt{\#!/bin/bash}
      \item \texttt{\#!/usr/bin/env ruby}
    \end{itemize}
  \end{itemize}
\end{frame}

\subsection{Funcionamiento}

\begin{frame}
  \frametitle{Pasaje de parámetros}
  \begin{itemize}
    \item Se los accede posicionalmente, a partir del índice \texttt{1}:
    \begin{itemize}
      \item \texttt{\$1} a \texttt{\$n}
      \item \texttt{\$0}
      \item \texttt{\$*}
      \item \texttt{\$\#}
    \end{itemize}
  \end{itemize}
\end{frame}

\begin{frame}
  \frametitle{Estado de los comandos - \textit{exit status}}
  \begin{itemize}
    \item Denotan si la operación realizada por un comando o una función
      fue exitosa o no
      \begin{itemize}
        \item \texttt{0} en caso de éxito
        \item \texttt{1} a \texttt{255} en caso de error
      \end{itemize}
    \item Se retornan con comandos internos:
    \begin{itemize}
      \item \texttt{exit} en el contexto de un comando/script
      \item \texttt{return} en el contexto de una función
    \end{itemize}
    \item Se consultan con una variable especial: \texttt{\$?}
  \end{itemize}
\end{frame}

\begin{frame}
  \frametitle{Sustitución de comandos}
  \begin{itemize}
    \item Reemplaza un comando por su salida estándar
    \item Literalmente coloca la salida de un comando en el lugar que
      este ocupa
    \item Se realiza con cadenas especiales:
    \begin{itemize}
      \item \texttt{`}: \texttt{echo \string"mi nombre es `whoami`\string"}
      \item \texttt{\$()}: \texttt{echo \string"mi nombre es \$(whoami)\string"}
    \end{itemize}
  \end{itemize}
\end{frame}

\begin{frame}
  \frametitle{Ejecución de scripts}
  \begin{itemize}
    \item Si se tienen permisos de ejecución:
    \begin{itemize}
      \item \texttt{./script.sh}
    \end{itemize}
    \item Especificando el binario a utilizar
    \begin{itemize}
      \item \texttt{bash script.sh}
    \end{itemize}
    \item En modo de depuración
    \begin{itemize}
      \item \texttt{bash -x script.sh}
    \end{itemize}
  \end{itemize}
\end{frame}

\subsection{En *nix todo es un archivo}

\begin{frame}
  \frametitle{En *nix todo es un archivo}
  \begin{itemize}
    \item Cada comando tiene \textit{al menos} 3 archivos abiertos:
    \begin{itemize}
      \item \texttt{stdin} (\texttt{0}): entrada estándar
      \item \texttt{stdout} (\texttt{1}): salida estándar
      \item \texttt{stderr} (\texttt{2}): salida de error estándar
    \end{itemize}
    \item Como son archivos, podemos leer de y escribir en ellos
    \begin{itemize}
      \item \texttt{read variable} lee de \texttt{stdin}
      \item \texttt{read variable < ARCHIVO} escribe el contenido de \texttt{ARCHIVO} en \texttt{stdin}
        para el comando \texttt{read}
      \item \texttt{echo Un mensaje} escribe en \texttt{stdout} (equivale a \texttt{echo Un mensaje >\&1})
      \item \texttt{echo Un mensaje >\&2} escribe en \texttt{stderr}
    \end{itemize}
  \end{itemize}
\end{frame}

\subsection{Comandos útiles}

\begin{frame}[fragile]
  \frametitle{Imprimir texto}
  \begin{itemize}
    \item Imprimir texto
   \begin{lstlisting}
 echo Shell scripting es sencillo
   \end{lstlisting}
  \end{itemize}
\end{frame}

\begin{frame}[fragile]
  \frametitle{Imprimir el contenido de un archivo}
  \begin{itemize}
    \item Imprimir el contenido de un archivo
   \begin{lstlisting}
 cat archivo
   \end{lstlisting}
  \end{itemize}
\end{frame}

\begin{frame}[fragile]
  \frametitle{Leer de teclado}
  \begin{itemize}
    \item Leer de entrada estándar (teclado) en una variable
   \begin{lstlisting}
 read variable
   \end{lstlisting}
  \end{itemize}
\end{frame}

\begin{frame}[fragile]
  \frametitle{Leer archivos estructurados}
  \begin{itemize}
    \item Leer el primer campo delimitado por :
   \begin{lstlisting}
 cut -d: -f1 # lee de stdin
 cut -d: -f1 archivo # lee desde archivo
   \end{lstlisting}
  \end{itemize}
\end{frame}

\begin{frame}[fragile]
  \frametitle{Traducir (intercambiar) caracteres}
  \begin{itemize}
    \item Traducir (intercambiar) caracteres
   \begin{lstlisting}
 echo hola | tr a-z A-Z # traduce las minúsculas por mayúsculas
   \end{lstlisting}
  \end{itemize}
\end{frame}

\begin{frame}[fragile]
  \frametitle{Contar la cantidad de líneas}
  \begin{itemize}
    \item Contar la cantidad de líneas
   \begin{lstlisting}
 wc -l # de stdin
 wc -l archivo # de archivo
   \end{lstlisting}
  \end{itemize}
\end{frame}

\begin{frame}[fragile]
  \frametitle{Buscar archivos}
  \begin{itemize}
    \item Buscar recursivamente archivos por su nombre en el directorio actual (\texttt{.})
   \begin{lstlisting}
 find . -name "*.sh"
   \end{lstlisting}
  \end{itemize}
\end{frame}

\begin{frame}[fragile]
  \frametitle{Buscar un patrón}
  \begin{itemize}
    \item Buscar un patrón
   \begin{lstlisting}
 grep shell # en stdin
 grep shell archivo # en archivo
 grep shell * # en todos los archivos del directorio actual
   \end{lstlisting}
  \end{itemize}
\end{frame}

\begin{frame}[fragile]
  \frametitle{Empaquetado}
  \begin{itemize}
    \item Empaquetado de archivos
   \begin{lstlisting}
 tar -cf destino.tar origen1 origen2 # empaqueta en destino.tar
 tar -xf destino.tar # desempaqueta destino.tar
   \end{lstlisting}
  \end{itemize}
\end{frame}

\begin{frame}[fragile]
  \frametitle{Compresión de archivos}
  \begin{itemize}
    \item Compresión de archivos
   \begin{lstlisting}
 gzip destino.tar # comprime destino.tar en destino.tar.gz
 gzip -d destino.tar.gz # descomprime destino.tar.gz
   \end{lstlisting}
  \end{itemize}
\end{frame}

\begin{frame}[fragile]
  \frametitle{Empaquetado y compresión con tar}
  \begin{itemize}
    \item Empaquetado y compresión con \texttt{tar}
   \begin{lstlisting}
 tar -cfz destino.tar.gz origen1 origen2 # invoca gzip automáticamente
 tar -xfz destino.tar.gz # descomprime y desempaqueta destino.tar.gz
   \end{lstlisting}
  \end{itemize}
\end{frame}

\begin{frame}[fragile]
  \frametitle{Listado de procesos}
  \begin{itemize}
    \item Listado de procesos
   \begin{lstlisting}
 ps -aux # lista los procesos en ejecución de todos los usuarios
   \end{lstlisting}
  \end{itemize}
\end{frame}

\subsection{Ejemplos}

\begin{frame}
  \frametitle{Ejemplos}
  \begin{itemize}
    \item https://github.com/unlp-so/ep3-shell-scripting
  \end{itemize}
\end{frame}
