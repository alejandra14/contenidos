\section{Módulos}

\begin{frame}{¿Que son los \textit{Módulos} del Kernel?}
  \begin{itemize}
  \item “Pedazos de código” que pueden ser cargados y descargados bajo demanda 
  \item Extienden la funcionalidad del kernel
  \item Sin ellos el kernel sería 100\% monolítico   
  \begin{itemize}
    \item Monolítico ``hibrido''
  \end{itemize} 
 \item No recompilar ni rebootear el kernel
  \end{itemize}
\end{frame}

\begin{frame}{Comandos relacionados}
  \begin{itemize}
  \item lsmod 
  	\begin{itemize}
  	  \item Lista los módulos cargados (es equivalente a cat /proc/modules) 
  	\end{itemize} 
  \item rmmod
  	\begin{itemize}
  	  \item Descarga uno o más módulos 
  	\end{itemize} 
  \item modinfo   
	  \begin{itemize}
	    \item  Muestra información sobre el módulo 
	  \end{itemize} 
 \item insmod
	  \begin{itemize}
	    \item Trata de cargar el módulo especificado 
	  \end{itemize} 
 \item depmod
	  \begin{itemize}
	    \item Permite calcular las dependencias de un módulo
             \item depmod -a escribe las dependencias en el archivo /lib/modules/version/modules.emp  
	  \end{itemize} 
 \item modprobe    
 	\begin{itemize}
             \item Emplea la información generada por depmod e información de /etc/modules.conf 
             para cargar el módulo especificado. 
  	\end{itemize} 
  \end{itemize}

\end{frame}

\begin{frame}[fragile]
\frametitle{¿Como creamos un módulo?}
  \begin{itemize}
  \item Debemos proveer dos funciones:
 	 \begin{itemize}
 	   \item Inicialización: Ejecutada cuando ejecutamos insmod.
 	 \end{itemize} 
	 \begin{itemize}
 	   \item Descarga: Ejecutada cuando ejecutamos rmmod.
 	 \end{itemize} 
  \end{itemize}

\lstset{language=C,
                basicstyle=\ttfamily,
                keywordstyle=\color{blue}\ttfamily,
                stringstyle=\color{red}\ttfamily,
                commentstyle=\color{green}\ttfamily,
                morecomment=[l][\color{magenta}]{\#}
}
\begin{lstlisting}
#include <linux/module.h> 
#include <linux/kernel.h> 
int init_module(void){
	printk(KERN_INFO "Hello world 1.\n");
return 0;
}

void cleanup_module(void){
	printk(KERN_INFO "Goodbye world 1.\n");
}
\end{lstlisting}
\end{frame}


\begin{frame}[fragile]
\frametitle{¿Como creamos un módulo?(cont.)}
  \begin{itemize}
  \item También podemos indicarle otras funciones.
 	 \begin{itemize}
 	   \item module\_init
	   \item module\_exit
 	 \end{itemize} 
  \end{itemize}

\lstset{language=C,
                basicstyle=\ttfamily,
                keywordstyle=\color{blue}\ttfamily,
                stringstyle=\color{red}\ttfamily,
                commentstyle=\color{green}\ttfamily,
                morecomment=[l][\color{magenta}]{\#}
}
\begin{lstlisting}
#include <linux/module.h>
#include <linux/kernel.h>
#include <linux/init.h>
static int hello_2_init(void){
	printk(KERN_INFO "Hello, world 2\n");
return 0;}

static void hello_2_exit(void){
printk(KERN_INFO "Goodbye, world 2\n");}

module_init(hello_2_init);
module_exit(hello_2_exit);
\end{lstlisting}
\end{frame}

\begin{frame}[fragile]
\frametitle{¿Como compilamos un módulo?}
  \begin{itemize}
  \item Construimos el Makefile
  \end{itemize}
\begin{lstlisting}
obj-m += hello.o
all:
make -C /lib/modules/$(shell uname -r)/build M=$(pwd) modules
clean:
make -C /lib/modules/$(shell uname -r)/build M=$(pwd) clean
\end{lstlisting}
  \begin{itemize}
  \item Compilamos
  \end{itemize}
\begin{lstlisting}
    $ make
\end{lstlisting}

\end{frame}

\begin{frame}{¿Que son los \textit{Dispositivos}?}
  \begin{itemize}
	\item Entendemos por dispositivo a cualquier dispositivo de hard: discos, memoria, mouse, etc
	\item Cada operación sobre un dispositivo es llevada por código específico para el dispositivo
	\item Este código se denomina “driver” y se implementa como un módulo
        \item Cada dispositivo de hardware es un archivo (abstracción)
 	\item Ejemplo: /dev/hda
	\begin{itemize}
		\item En realidad no es un archivo.
		\item Si leemos/escribimos desde él lo hacemos sobre datos “crudos” del disco (bulk data).
	\end{itemize}
         \item Accedemos a estos archivos mediante operaciones básicas (espacio del kernel).
	\begin{itemize}
		\item read, write: escribir y recuperar bulk data
		\item ioctl: configurar el dispositivo 

	\end{itemize}
  \end{itemize}
\end{frame}


\begin{frame}{¿Que son los \textit{Dispositivos}? (Cont.)}
  \begin{itemize}
	\item Podemos clasificar el hard en varios tipos.
	\begin{itemize}
		\item Dispositivos de acceso aleatorio(ej. discos). 
		\item Dispositivos seriales(ej. Mouse, sonido,etc).
	\end{itemize}	
	\item Acorde a esto los drivers se clasifican en:
     	\begin{itemize}
		\item Drivers de bloques: son un grupo de bloques de datos persistentes. Leemos y escribimos de a bloques, generalmente de 			1024 bytes. 
		\item Drivers de carácter: Se accede de a 1 byte a la vez y 1 byte sólo puede ser leído por única vez.
		\item Drivers de red: tarjetas ethernet, WIFI, etc.
	\end{itemize}	
  \end{itemize}
\end{frame}

\begin{frame}[fragile]
\frametitle{\textit{Dispositivos} en UNIX}  
  \begin{itemize}
	\item Major y Minor device number.
	\begin{itemize}
		\item Los dispositivos se dividen en números llamados major device number. Ej: los discos SCSI tienen el major number 8.
		\item Cada dispositivo tiene su minor device number. Ejemplo: /dev/sda major number 8 y minor number 0 
	\end{itemize}	
	\item Con el major y el minor number el kernel identifica un dispositivo.
	\item kernel\_code/linux/Documentation/devices.txt
  \end{itemize}
\begin{lstlisting}
# ls -l /dev/hda[1-3]
brw-rw---- 1 root disk 3, 1 Abr 9 15:24 /dev/hda1
brw-rw---- 1 root disk 3, 2 Abr 9 15:24 /dev/hda2
brw-rw---- 1 root disk 3, 3 Abr 9 15:24 /dev/hda3
\end{lstlisting}
\end{frame}

\begin{frame}[fragile]
\frametitle{Archivos de dispositivos en UNIX}  
  \begin{itemize}
	\item Representación de los dispositivos(device files)
	\item Por convención están en el /dev
	\item Se crean mediante el comando mknod.
  \end{itemize}
\begin{lstlisting}
mknod [-m <mode>] file [b|c] major  minor
\end{lstlisting}
  \begin{itemize}
	\item b o c: según se trate de dispositivos de caracter o de bloque.
	\item El minor y el major number lo obtenemos de kernel\_code/linux/Documentation/devices.txt
  \end{itemize}
\end{frame}

\begin{frame}{Dispositivos en UNIX}
  \begin{itemize}
  \item Necesitamos decirle al kernel: 
  \begin{itemize}
    \item Que hacer cuando se escribe al device file.
    \item Que hacer cuando se lee desde el device file..
  \end{itemize} 
 \item Todo esto lo hacemos en un módulo.
 \item La struct \alert{file\_operations}: 
  \begin{itemize}
    \item Sirve para decirle al kernel como leer y/o escribir al dispositivo.
    \item Cada variable posee un puntero a las funciones que implementan las operaciones sobre el dispositivo.
  \end{itemize}   

  \end{itemize}
\end{frame}

\begin{frame}[fragile]
\frametitle{¿Como creamos un driver?}
  \begin{itemize}
  \item Mediante la struct \alert{file\_operations} especifico que funciones leen/escriben al dispositivo. 
  \end{itemize}
\lstset{language=C,
                basicstyle=\ttfamily,
                keywordstyle=\color{blue}\ttfamily,
                stringstyle=\color{red}\ttfamily,
                commentstyle=\color{green}\ttfamily,
                morecomment=[l][\color{magenta}]{\#}
}
\begin{lstlisting}
struct file_operations my_driver_fops = {
read: myDriver_read,
write: myDriver_write,
open: myDriver_open,
release: mydriver_release};
\end{lstlisting}
\begin{itemize}
 \item En la función module\_init registro mi driver. 
\end{itemize}
\lstset{language=C,
                basicstyle=\ttfamily,
                keywordstyle=\color{blue}\ttfamily,
                stringstyle=\color{red}\ttfamily,
                commentstyle=\color{green}\ttfamily,
                morecomment=[l][\color{magenta}]{\#}
}
\begin{lstlisting}
register_chrdev(major_number, "myDriver", &my_driver_fops);
\end{lstlisting}

\begin{itemize}
 \item En la función module\_exit desregistro mi driver. 
\end{itemize}
\lstset{language=C,
                basicstyle=\ttfamily,
                keywordstyle=\color{blue}\ttfamily,
                stringstyle=\color{red}\ttfamily,
                commentstyle=\color{green}\ttfamily,
                morecomment=[l][\color{magenta}]{\#}
}
\begin{lstlisting}
unregister_chrdev(major_number, "myDriver");
\end{lstlisting}
\end{frame}

\begin{frame}[fragile]
\frametitle{¿Como creamos un driver?(Cont..)}
  \begin{itemize}
  \item Operaciones sobre el dispositivo  
  \begin{itemize}
    \item Escritura del archivo de dispositivo (Ej. echo ``hi'' > /dev/myDeviceFile) 
  \end{itemize}	
  \end{itemize}
\lstset{language=C,
                basicstyle=\ttfamily,
                keywordstyle=\color{blue}\ttfamily,
                stringstyle=\color{red}\ttfamily,
                commentstyle=\color{green}\ttfamily,
                morecomment=[l][\color{magenta}]{\#}
}
\begin{lstlisting}
ssize_t myDriver_write(struct file *filp, char *buf,size_t count, loff_t *f_pos);
\end{lstlisting}

\begin{itemize}
  \item Lectura del archivo de dispositivo  (Ej. cat /dev/myDeviceFile)  
 \end{itemize}

\lstset{language=C,
                basicstyle=\ttfamily,
                keywordstyle=\color{blue}\ttfamily,
                stringstyle=\color{red}\ttfamily,
                commentstyle=\color{green}\ttfamily,
                morecomment=[l][\color{magenta}]{\#}
}
\begin{lstlisting}
ssize_t myDriver_read(struct file *filp, char *buf,size_t count, loff_t *f_pos)
\end{lstlisting}
\end{frame}

\begin{frame}{¿Como creamos un driver? (cont.)}
  \begin{itemize}
  \item Parámetros de las funciones funciones:
 	 \begin{itemize}
 	   \item Struct file: Estructura del kernel que representa un archivo abierto.
	   \item char *buf: El dato a leído o a escribir desde/hacia el dispositivo(espacio de usuario) 
	   \item size\_t count: La longitud de buf. 
           \item loff\_t *f\_pos count: La posición manipulada
 	 \end{itemize}  
  \end{itemize}
\end{frame}

\begin{frame}
\frametitle{EXPERIMENTO - Explicación}
   \begin{itemize}  
   \item Implementar un módulo que encienda los leds del teclado en base a un número escrito en una entrada del \emph{/proc}.
   \item Cadena puede incluir los caracteres ‘1’,‘2’ y ‘3’:
   \begin{itemize}
      \item Si aparece el ‘1’ → encender Num Lock
      \item Si aparece el ‘2’ → encender Caps Lock
      \item Si aparece el ‘3’ → encender Scroll Lock
    \end{itemize}
   \item Tips:
   \begin{itemize}
      \item \emph{procfs} (\textit{process filesystem})
      \item File Operations      
      \item KBD Driver
    \end{itemize}
 \end{itemize}
\end{frame}

\begin{frame}[fragile]
\frametitle{EXPERIMENTO - Implemetación}
  \begin{itemize}
    \item Utilizar:
       \begin{itemize}
          \item \href{https://github.com/unlp-so/contenidos/tree/master/explicaciones/so}{Fichero .c}
          \item \href{https://github.com/unlp-so/contenidos/tree/master/explicaciones/so}{Makefile}
       \end{itemize}
  \end{itemize}
  \begin{lstlisting}
#Compilamos el módulo
$ make
  \end{lstlisting}
\end{frame}

\begin{frame}[fragile]
\frametitle{EXPERIMENTO - Test}
  \begin{lstlisting}
#Insertar el módulo
$ sudo insmod procleds.ko

#Encender leds
$ sudo echo 1 > /proc/leds
$ sudo echo 123 > /proc/leds

#Eliminar módulo
$ sudo rmmod procleds
  \end{lstlisting}
\end{frame}